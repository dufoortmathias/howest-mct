\documentclass{article}

\usepackage[dutch]{babel}
\usepackage[margin=3cm]{geometry}
\usepackage{graphicx}
\usepackage{float}
\usepackage{caption}
\usepackage{hyperref}
\usepackage{amsmath}
\usepackage{wrapfig}
\usepackage[parfill]{parskip}

% fonts
\usepackage[T1]{fontenc}
\usepackage{helvet}
\renewcommand{\familydefault}{\sfdefault}

\graphicspath{{img/}} 
 
\newcommand{\bold}[1]{\textbf{#1}}

\begin{document}

\begin{titlepage}
    \author{Tuur Vanhoutte}
    \title{Sensors \& Interfacing}
\end{titlepage}

\pagenumbering{gobble}
\maketitle
\newpage
\tableofcontents
\newpage

\pagenumbering{arabic}

\section {Communicatie}
\subsection{Datacommunicatie in IoT}
\subsubsection{De 3 lagen}
\begin{enumerate}
    \item Application Layer
    \item Fog layer
    \item IoT Device Layer
\end{enumerate}


\begin{figure}[H]
    \centering
    \includegraphics[width=0.8\textwidth]{Screenshot_20200210_120010.png}
    \caption{Datacommunicatie in IoT}
\end{figure}



\subsection{Data}
\begin{itemize}
    \item ``Pre-informatie''
    \item Gegevens waaruit informatie kan worden gewonnen
    \item Stelt een bepaalde toestand voor
    \item \url{https://en.wikipedia.org/wiki/Data}
\end{itemize}

\subsection{Communicatie}
= Overbrengen van informatie tussen deelnemers
\begin{itemize}
    \item Boodschap
    \item Signaal
    \item Medium
\end{itemize}

\subsection{Coderen van informatie (encoding)}
\subsubsection{Voorbeelden}
\begin{itemize}
    \item morse-code
    \item Ascii-codering
    \begin{itemize}
        \item Codering voor alle gebruikte symbolen in symbolen
        \item Codering in 7 of 8 bit
        \item 1 byte = 1 teken
    \end{itemize}
    \item \dots
\end{itemize}
  
\subsubsection{Encoding/Decoding}
3 stappen:
\begin{enumerate}
    \item Codifying
    \item Sending the message
    \item Decodifying
\end{enumerate}

\subsection{Signalen}
\begin{itemize}
    \item Licht
    \item Geluid
    \item Elektriciteit
    \item \dots
\end{itemize}

\subsection{Communicatiemedia}
\begin{itemize}
    \item Twisted-Pair cable
    \item Coaxial cable
    \item Fiber-Optic cable
\end{itemize}

\begin{figure}[H]
    \centering
    \includegraphics[width=\textwidth]{Screenshot_20200315_105647.png}
    \caption{Soorten kabels}
\end{figure}


\subsubsection{Eigenschappen van media}
\begin{itemize}
    \item Vatbaarheid voor interferentie
    \item Overbrugbare afstand
    \item Praktisch
    \item Kostprijs
    \item \dots
\end{itemize}

\subsubsection{Afspraken}
\begin{itemize}
    \item Protocol
    \item Standaarden
    \begin{itemize}
        \item Type media en zijn specificaties
        \item Het gebruikte signaal en zijn toleranties
        \item De elektrische interferentie
        \item De gebruikte codering
        \item Foutcorrectiecodes
        \item Protocol
        \item De gebruikte connector
        \item \dots
    \end{itemize}
    \item Gebeurt door IEEE
\end{itemize}

\section{Analoog vs digitaal}
\begin{itemize}
    \item \bold{Digitaal}: Discrete waarden
    \item \bold{Analoog}: Continue waarden
\end{itemize}
\subsection{Toestanden}
\subsubsection{Digitale toestanden}
\begin{itemize}
    \item Licht aan/uit
    \item Deur open/dicht
    \item Keuze van versnelling N - 1 - 2 - 3 - 4 - 5 - R
    \item Ruitenwisser interval uit - interval - traag - snel
    \item \dots
\end{itemize}

\subsubsection{Analoge toestanden}
\begin{itemize}
    \item Tijd (!)
    \item Temperatuur
    \item Luchtdruk
    \item Luchtvochtigheid
    \item Afstand
    \item \dots
\end{itemize}


\subsection{Signalen}
\begin{itemize}
    \item Analoog signaal
    \item Digitaal signaal
\end{itemize}

\begin{figure}[H]
    \centering
    \includegraphics[width=0.4\textwidth]{Screenshot_20200217_115642.png}
    \caption{Analoog vs digitaal signaal}
\end{figure}


\section{Analoge signalen}
\subsection{Omzetten van analoge signalen}
\subsubsection{Transducer}
Omzetten van een analoog signaal naar een ander analoog signaal.

\bold{Voorbeeld}: elektrisch signaal omzetten naar een geluidsignaal via een luidspreker (=de transducer)

\begin{figure}[H]
    \centering
    \includegraphics[width=0.4\textwidth]{Screenshot_20200315_111132.png}
    \caption{Luidspreker}
\end{figure}

\subsubsection{Sensoren en Actuatoren}
\begin{itemize}
    \item Sensor $\Rightarrow$ meten van een fysieke eigenschap
    \item Actuator $\Rightarrow$ be"invloeden van een fysieke parameter, bv transducers 
\end{itemize}

\subsection{Analoge communicatie}

\begin{figure}[H]
    \centering
    \includegraphics[width=0.5\textwidth]{Screenshot_20200315_111720.png}
    \caption{Analoge communicatie}
\end{figure}

\subsection{Analoog signaal}
Sinusgolf als meest elementaire signaal

\begin{figure}[H]
    \centering
    \includegraphics[width=0.5\textwidth]{Screenshot_20200315_113310.png}
    \caption{Sinusgolf}
\end{figure}

\subsubsection{Eigenschappen}
\begin{itemize}
    \item DC vs AC
    \item Polariteit blijft gelijk bij (pulserende) DC
    \item Polariteit verandert bij AC
\end{itemize}



\begin{figure}[H]
    \centering
    \includegraphics[width=0.5\textwidth]{Screenshot_20200217_121011.png}
    \caption{DC vs AC}
\end{figure}

\subsubsection{Wisselspanning - Eigenschappen}
\begin{itemize}
    \item RMS = Root Mean Square (= kwadratisch gemiddelde) = effectieve waarde (in geval van sinus)
    \begin{enumerate}
        \item Som van alle kwadraten (= square)
        \item Die som delen door het aantal waardes (= mean)
        \item Neem de vierkantswortel van dat getal
    \end{enumerate}
    \begin{itemize}
        \item Wordt vaak gebruikt in de elektriciteit om het gemiddelde vermogen te vinden
    \end{itemize}
    \item Frequentie
    \item Periode
    \item Amplitude
    \item Peak of top-to-top waarde
\end{itemize}

\begin{figure}[H]
    \centering
    \includegraphics[width=0.8\textwidth]{Screenshot_20200217_121136.png}
    \caption{Eigenschappen wisselspanning}
\end{figure}

\subsubsection{Periodieke signalen}
\begin{itemize}
    \item 1 herhaling = 1 periode
    \item Periode (T) = tijdsduur (in s)
    \item Frequentie (f) = aantal periodes per seconde (in Hz)
    \item $F = \frac1T$ $\Leftrightarrow$ $T = \frac1F$
\end{itemize}

\begin{figure}[H]
    \centering
    \includegraphics[width=0.4\textwidth]{Screenshot_20200217_122002.png}
    \caption{Sinusgolf met periode $T$}
\end{figure}


\subsubsection{Tijdsdomein en frequentiedomein}
\begin{itemize}
    \item Tijdsdomein: met een oscilloscoop. Toont de amplitude over de tijd.
    \item Frequentiedomein: met spectraalanalyse. Toont de amplitude over de frequentie.
\end{itemize}

\begin{figure}[H]
    \centering
    \centerline{
        \includegraphics[width=0.4\textwidth]{Screenshot_20200217_122108.png}
        \includegraphics[width=0.5\textwidth]{Screenshot_20200217_122246.png}    
    }
    \caption{Tijdsdomein vs Frequentiedomein}
\end{figure}

\section{Digitale signalen}
= Aan/uit

\subsection{Eigenschappen}
\begin{itemize}
    \item Amplitude: Piek- of top-waarde, RMS-waarde
    \item Periode / frequentie
    \item Pulsbreedte 
    \item Duty-cycle
\end{itemize}


\begin{figure}[H]
    \centering
    \includegraphics[width=0.6\textwidth]{Screenshot_20200315_120112.png}
    \caption{Eigenschappen digitaal signaal}
\end{figure}

\subsection{Duty Cycle}
= Hoeveel procent van de tijd staat het signaal aan?

\begin{figure}[H]
    \centering
    \includegraphics[width=0.5\textwidth]{Screenshot_20200315_120616.png}
    \caption{Duty cycles}
\end{figure}

\subsection{Flanken (edge)}
\begin{itemize}
    \item Stijgende flank
    \item Dalende flank
    \item Belangrijk bij kloksignalen
\end{itemize}

\begin{figure}[H]
    \centering
    \includegraphics[width=0.5\textwidth]{Screenshot_20200217_123230.png}
    \caption{Flanken}
\end{figure}

\subsection{Weergave digitale signalen}

\begin{figure}[H]
    \centering
    \includegraphics[width=0.8\textwidth]{Screenshot_20200217_125726.png}
    \caption{Weergave digitale signalen}
\end{figure}


\section{AD conversie}

\begin{figure}[H]
    \centering
    \includegraphics[width=0.7\textwidth]{Screenshot_20200224_115204.png}
    \caption{Analoog naar digitaal (AD converter)}
\end{figure}

\subsection{Analoog naar digitaal}

\begin{figure}[H]
    \centering
    \includegraphics[width=0.7\textwidth]{Screenshot_20200224_115430.jpg}
    \caption{Analoog naar digitaal conversie}
\end{figure}

\begin{itemize}
    \item \bold{Range} = verschil tussen laagste en hoogste waarde
    \item \bold{Resolutie} = aantal stappen of stapgrootte in bits
    \item \bold{Belangrijk gevolg:}
    \begin{itemize}
        \item beide parameters bepalen de exactheid en de afwijkingen
    \end{itemize}
\end{itemize}

\subsubsection{Voorbeeld A}
\begin{itemize}
    \item Range = 2V - 2.5V
    \item Resolutie = 8bits
    \item Dus aantal discrete stappen = $2^8 = 256 \Rightarrow 256 - 1 = 255$
    \item Stapgrootte (LSB) = $\frac{range}{255} = \frac{2.5V - 2V}{255} = \frac{0.5V}{255} = 0.00196..V / stap$
    \item ofwel $\approx$ $2mV$/stap
\end{itemize}


\subsubsection{Voorbeeld B}
\begin{itemize}
    \item Range = 0V - 12V
    \item Resolutie = 12bits
    \item Dus aantal discrete stappen = $2^{12} = 4096 \Rightarrow 4096 - 1 = 4095$
    \item Stapgrootte (LSB) = $\frac{range}{4095} = \frac{12V - 0V}{4095} = \frac{12V}{4095} = 0.0029304..V / stap$
    \item ofwel $\approx$ $3mV$/stap
\end{itemize}

\subsection{Eigenschappen}

\subsubsection{Quantisatiefouten}

\begin{figure}[H]
    \centering
    \centerline{
        \includegraphics[width=0.5\textwidth]{Screenshot_20200224_120800.jpg}
        \includegraphics[width=0.5\textwidth]{Screenshot_20200224_120900.jpg}    
    }
    \caption{Quantisatiefouten door AD conversie}
\end{figure}

\begin{itemize}
    \item Verzorzaakt quantisatieruis
    \item Quantisatiefouten worden opgelost met dithering (= \underline{vooraf} (witte) ruis toevoegen aan signaal)
\end{itemize}

\begin{figure}[H]
    \centering
    \centerline{
        \includegraphics[width=0.4\textwidth]{Screenshot_20200224_121151.png}
        \includegraphics[width=0.3\textwidth]{Screenshot_20200315_122540.png}
    }
    \caption{Dithering}
\end{figure}

\subsubsection{Sample Rate / sample frequentie}
= aantal conversies per seconde
\begin{figure}[H]
    \centering
    \includegraphics[width=0.7\textwidth]{Screenshot_20200224_121653.png}
    \caption{Sample rate}
\end{figure}

\begin{itemize}
    \item \bold{Nyquist rate} = Minimale sample rate = 2x de frequentie van het signaal
    \item Voorbeelden:
    \begin{itemize}
        \item HiFi Audio CD: 44.1kHz sample rate
        \item Oude telefoontoestellen: 8kHz sample rate
        \item HD-DVD Audio: 192kHz
    \end{itemize}
\end{itemize}

\begin{figure}[H]
    \centering
    \includegraphics[width=0.6\textwidth]{Screenshot_20200224_121935.png}
    \caption{Minimale sample rate}
\end{figure}

\subsubsection{Aliasing}
= High Frequency signaal als Low Frequency 'spooksignaal' detecteren 
\begin{itemize}
    \item Treedt op bij onvoldoende hoge sample rate
    \item Anti-Aliasing filter (low-pass filter) beperkt signaal onder nyquist frequentie
\end{itemize}

\begin{figure}[H]
    \centering
    \includegraphics[width=0.7\textwidth]{Screenshot_20200224_122204.png}
    \caption{Anti-aliasing filter}
\end{figure}

\subsubsection{Oversampling}
\begin{itemize}
    \item Sampelen met veelvoud van nyquist frequentie
    \item Kan worden gebruikt om de resolutie op te voeren
    \item Kan worden gebruikt om digitaal (DSP) te filteren
    \item Verhoogt het effectieve aantal bits van de ADC
    \begin{itemize}
        \item \underline{Voorbeeld:} 20bit ADC met 256x OS = 24bit effectieve resolutie
    \end{itemize}
    \item Undersampling $\rightarrow$ specifiek gebruik bij mixers
\end{itemize}

\subsection{Implementatie en types}
\begin{itemize}
    \item De comparator
    \item Bekeken als 1-bit ADC
\end{itemize}

\begin{figure}[H]
    \centering
    \includegraphics[width=0.7\textwidth]{Screenshot_20200224_122504.png}
    \caption{Comparator}
\end{figure}

\subsubsection{Flash ADC}
\begin{itemize}
    \item Comparator per 'level'
    \item Zeer snel = directe omzetting
    \item Complex \& High power
    \item Lagere resoluties
\end{itemize}

\begin{figure}[H]
    \centering
    \includegraphics[width=0.5\textwidth]{Screenshot_20200224_122635.png}
    \caption{Flash ADC}
\end{figure}

\subsubsection{Successive approximation ADC}
\begin{itemize}
    \item Gebruikt 1 comparator
    \item Vergelijkt een opgewekte spanning met het signaal
    \item Hoge resolutie mogelijk
    \item Trager
    \item Relatief goedkoop
\end{itemize}

\begin{figure}[H]
    \centering
    \includegraphics[width=0.4\textwidth]{Screenshot_20200224_122846.png}
    \caption{Successive approximation ADC}
\end{figure}

\section{Digitaal naar analoog conversie}
\begin{figure}[H]
    \centering
    \includegraphics[width=0.7\textwidth]{Screenshot_20200224_115212.png}
    \caption{Digitaal naar analoog (DA converter)}
\end{figure}

\begin{itemize}
    \item Omzetten digitale naar analoge waarde
    \item Range
    \item Resolutie
    \item Samplefrequentie
\end{itemize}

\subsection{Simpele DAC}
\begin{itemize}
    \item Binaire waarde naar analoge waarde
    \item \underline{Voorbeeld}: weerstandsnetwerk
\end{itemize}

\begin{figure}[H]
    \centering
    \includegraphics[width=0.4\textwidth]{Screenshot_20200224_123043.png}
    \caption{Weerstandsnetwerk}
\end{figure}

\subsection{Simpele DAC met PWM}
\begin{itemize}
    \item PWM == digitaal signaal
    \item Door variatie van duty-cycle kan de gemiddelde waarde worden gevarieerd
    \item Door filteren kan de blokgolf worden omgezet in een variabele analoge waarde 
\end{itemize}

\begin{figure}[H]
    \centering
    \includegraphics[width=0.7\textwidth]{Screenshot_20200224_123240.png}
    \caption{DAC met PWM}
\end{figure}

\subsection{Andere types}
\begin{itemize}
    \item $\Sigma\Delta$ (sigma delta) = herhaaldelijk downsampelen
    \item $I^2S$ DAC
    \item Nog zeer veel andere overwegingen:
    \begin{itemize}
        \item THD (Harmonische vervorming)
        \item Faseruis
        \item \dots
    \end{itemize}
\end{itemize}

\subsection{Extra info over AD/DA conversie:}
\begin{itemize}
    \item \url{https://en.wikipedia.org/wiki/Digital-to-analog_converter}
    \item \url{https://en.wikipedia.org/wiki/Analog-to-digital_converter}
    \item \url{https://en.wikipedia.org/wiki/Nyquist%E2%80%93Shannon_sampling_theorem}
    \item \url{https://en.wikipedia.org/wiki/Nyquist_rate}
    \item \url{https://en.wikipedia.org/wiki/Pulse-width_modulation}
    \item \url{https://en.wikipedia.org/wiki/Delta-sigma_modulation}
\end{itemize}

\section{Modulatie}
\begin{itemize}
    \item Informatie toevoegen aan een draaggolf
    \item Door de variatie van minstens een van de eigenschappen van deze draaggolf
\end{itemize}

\begin{figure}[H]
    \centering
    \includegraphics[width=0.6\textwidth]{Screenshot_20200302_115326.png}
    \caption{Modulatie}
\end{figure}

\subsection{Demodulatie}
= Het terugwinnen van de informatie uit de gemoduleerde draaggolf

\begin{figure}[H]
    \centering
    \includegraphics[width=0.7\textwidth]{Screenshot_20200302_115359.png}
    \caption{Demodulatie}
\end{figure}

\subsection{Modem}
= \bold{Mo}dulator + \bold{Dem}odulator

\begin{figure}[H]
    \centering
    \includegraphics[width=0.3\textwidth]{Screenshot_20200302_115504.png}
    \caption{Modem}
\end{figure}

\subsection{Waarom?}

\begin{itemize}
    \item Interconnectie van IoT devices
    \item Vaak draadloos
\end{itemize}

\begin{figure}[H]
    \centering
    \includegraphics[width=0.5\textwidth]{Screenshot_20200302_115646.png}
    \caption{Interconnectie van devices}
\end{figure}

\subsection{Hoe?}
\begin{itemize}
    \item Draaggolf of carrier
    \item Signaal met een zekere (hogere) frequentie
\end{itemize}

\subsection{Draaggolf}
= carrier

\begin{figure}[H]
    \centering
    \includegraphics[width=0.8\textwidth]{Screenshot_20200302_115827.png}
    \caption{Draaggolf}
\end{figure}

\subsubsection{Parameters van een draaggolf}

\begin{itemize}
    \item Amplitude
    \item Frequentie / Periode
\end{itemize}

\subsection{Simpele modulatie}
\begin{itemize}
    \item Aan/uit schakelen van de draaggolf
    \item CW = continuous wave
    \item Bv: Morse code
\end{itemize}

\begin{figure}[H]
    \centering
    \includegraphics[width=0.7\textwidth]{Screenshot_20200302_120000.png}
    \caption{Morse code}
\end{figure}

\subsection{Amplitude modulatie (=AM)}
\begin{itemize}
    \item Aanpassen van de amplitude v/d draaggolf
    \item Radio LW/MW/SW AM
    \item Typisch gebruikt op 'lagere' HF banden: 100kHz - 60MHz
\end{itemize}

\begin{figure}[H]
    \centering
    \includegraphics[width=0.5\textwidth]{Screenshot_20200302_120121.png}
    \includegraphics[width=0.6\textwidth]{Screenshot_20200302_120147.png}
    \caption{Amplitudemodulatie}
\end{figure}
    

\subsubsection{Overmodulatie}
= Als de modulatiediepte (=modulatie index) groter dan 100\% wordt.

\begin{figure}[H]
    \centering
    \includegraphics[width=0.8\textwidth]{Screenshot_20200302_120354.png}
    \caption{Overmodulatie bij 150\% modulatiediepte}
\end{figure}

\subsubsection{AM bandbreedte}
\begin{itemize}
    \item Centerfrequentie = draaggolffrequentie
    \item 2x frequentie van gemoduleerd signaal in totaal
\end{itemize}

\begin{figure}[H]
    \centering
    \includegraphics[width=0.8\textwidth]{Screenshot_20200302_120508.png}
    \caption{AM bandbreedte van 8000Hz}
\end{figure}

\subsubsection{SSB modulatie (USB/LSB)}
= Single-SideBand modulatie (Upper / Lower SideBand)
\begin{itemize}
    \item Alle informatie zit in elke sideband (zijband) bij AM
    \item Carrier + 1 sideband wegfilteren = reductie van de bandbreedte
    \item Efficienter gebruik van het spectrum
    \item Moeilijk om goed te demoduleren
\end{itemize}
\begin{figure}[H]
    \centering
    \includegraphics[width=0.7\textwidth]{Screenshot_20200315_153734.png}
    \caption{Zijbanden AM band}
\end{figure}

\begin{figure}[H]
    \centering
    \includegraphics[width=\textwidth]{Screenshot_20200302_120812.png}    
    \caption{SSB Modulatie: we zenden alleen 1 van de zijbanden, zonder de carrier}
\end{figure}

\subsubsection{ASK Modulatie}
= Amplitude Shift Keying modulatie
\begin{itemize}
    \item Vorm van AM-modulatie voor digitale signalen
    \item Mogelijk met meerdere signaalniveaus (`levels')
\end{itemize}

\begin{figure}[H]
    \centering
    \includegraphics[width=0.6\textwidth]{Screenshot_20200302_121007.png}
    \caption{ASK}
\end{figure}

\begin{figure}[H]
    \centering
    \includegraphics[width=0.6\textwidth]{Screenshot_20200302_121031.png}
    \caption{Meerdere signaalniveau's: 4 level ASK}
\end{figure}

\subsection{Frequentie modulatie (=FM)}
\begin{itemize}
    \item Variatie in de frequentie van de draaggolf
    \item FM radio 88MHz - 108MHz
    \item VHF Maritieme radio
    \item UHF PMR Radios
\end{itemize}

\begin{figure}[H]
    \centering
    \includegraphics[width=0.5\textwidth]{Screenshot_20200302_121335.png}
    \caption{Frequentie modulatie}
\end{figure}

\subsubsection{FSK modulatie}
= Frequency Shift Keying modulatie

= Vorm van FM waarbij gewisseld wordt tussen 2 of meer frequenties

\begin{figure}[H]
    \centering
    \includegraphics[width=0.6\textwidth]{Screenshot_20200302_121458.png}
    \caption{FSK}
\end{figure}

\subsubsection{FM Modulatie}
\begin{itemize}
    \item Carrier is altijd op 100\% amplitude aanwezig tijdens transmissie
    \item Minder ruis, betere kwaliteit voor audio dan AM
    \item Hogere bandbreedte $\Rightarrow$ Meer stroomverbruik
    \item WFM (Wideband FM), NFM (Narrowband FM), FM
\end{itemize}

\begin{figure}[H]
    \centering
    \includegraphics[width=0.8\textwidth]{Screenshot_20200302_121650.png}
    \caption{Bandbreedte FM - NFM - WFM}
\end{figure}

\subsubsection{Fase modulatie}
\begin{itemize}
    \item Faseverschuiving van een signaal
\end{itemize}

\begin{figure}[H]
    \centering
    \includegraphics[width=0.7\textwidth]{Screenshot_20200302_121855.png}
    \caption{Faseverschuiving}
\end{figure}

\subsubsection{PSK modulatie}
\begin{itemize}
    \item Phase shift keying
    \item Bij wisselen van bit $\Rightarrow$ fase omkeren
\end{itemize}

\begin{figure}[H]
    \centering
    \includegraphics[width=0.6\textwidth]{Screenshot_20200302_122031.png}
    \caption{Fasemodulatie}
\end{figure}

\subsubsection{Phase Shift Modulatie}
\begin{itemize}
    \item BPSK $\Rightarrow$ Binary PSK = 2 fase
    \item QPSK $\Rightarrow$ Quadrature PSK = 4 fase
\end{itemize}

\begin{figure}[H]
    \centering
    \includegraphics[width=0.6\textwidth]{Screenshot_20200302_122452.png}
    \caption{QPSK}
\end{figure}

\subsubsection{QAM modulatie}
\begin{itemize}
    \item Quadrature amplitude modulatie
    \item Informatie zit in \dots
    \begin{itemize}
        \item de amplitude (zoals bij ASK)
        \item de fase (zoals bij PSK)
    \end{itemize}
    \item Meerdere vormen
    \begin{itemize}
        \item 4-QAM
        \item 16-QAM
        \item 64-QAM
        \item \dots
    \end{itemize}
\end{itemize}

\begin{figure}[H]
    \centering
    \includegraphics[width=0.3\textwidth]{Screenshot_20200315_160021.png}
    \caption{16QAM modulatie}
\end{figure}


\begin{itemize}
    \item DAB+ = 4/16/64-QAM
    \item Hogere transmissiesnelheid
    \item Gevoeliger voor fouten
\end{itemize}

\begin{figure}[H]
    \centering
    \includegraphics[width=0.7\textwidth]{Screenshot_20200302_123030.png}
    \caption{16/64/256-QAM}
\end{figure}

\subsubsection{Bandbreedte / vermogen}
Meer bandbreedte = hogere snelheid, maar ook hoger vermogen nodig


\begin{figure}[H]
    \centering
    \includegraphics[width=0.6\textwidth]{Screenshot_20200302_122703.png}
    \caption{Bandbreedte/vermogen verdeling}
\end{figure}


\subsubsection{Waarom Narrow-band?}
\begin{itemize}
    \item Oppervlakte van het signaal = Power
    \item Bij smalbandige signalen $\Rightarrow$ betere SNR (signaal-naar-ruis) verhouding bij hetzelfde vermogen
    \item Om met een klein vermogen heel grote afstanden overbruggen
    \item Zeer traag
\end{itemize}

\begin{figure}[H]
    \centering
    \includegraphics[width=0.6\textwidth]{Screenshot_20200302_123310.png}
    \caption{Waarom Narrow-band: narrow-band kan makkelijker boven ruisniveau}
\end{figure}

\begin{figure}[H]
    \centering
    \includegraphics[width=0.8\textwidth]{Screenshot_20200302_122906.png}
    \caption{Voorbeeld: Narrowband-IoT}
\end{figure}

\begin{figure}[H]
    \centering
    \includegraphics[width=0.4\textwidth]{Screenshot_20200302_122954.png}
    \caption{Voorbeeld: Telefoonmodem (\url{https://www.youtube.com/watch?v=ckc6XSSh52w})}
\end{figure}

\section{RF-spectrum}

\begin{figure}[H]
    \centering
    \includegraphics[width=\textwidth]{Screenshot_20200309_115132.png}
    \caption{Electromagnetisch spectrum: radiogolven}
\end{figure}

\subsection{Beschikbare bandbreedte}
\begin{itemize}
    \item In de hogere frequentiebanden is meer bandbreedte beschikbaar
    \item \underline{Voorbeelden:}
    \begin{itemize}
        \item 2.4GHz WiFi: 20MHz bandbreedte
        \item 5GHZ 802.11ac WiFi: 80MHz bandbreedte
        \item 1 WiFi kanaal $>$ volledige AM-radio LW + MW + SW band
    \end{itemize}
    \item Meer bandbreedte = mogelijk hogere datarates
\end{itemize}

\begin{figure}[H]
    \centering
    \includegraphics[width=0.4\textwidth]{Screenshot_20200315_160828.png}
    \caption{Impact van hogere bandbreedtes op datarates}
\end{figure}

\subsection{Licenties en licentievrije banden}
\begin{itemize}
    \item Beheer van het spectrum:
    \begin{itemize}
        \item In Belgie door het \bold{BIPT}
        \item In de VS door het \bold{FCC}
        \item Harmonisatie in Europa / VS / Azië
    \end{itemize}
    \item \url{https://www.bipt.be/nl/operatoren/radio/frequentiebeheer/frequentieplan/tabel}
\end{itemize}

\subsection{ISM banden}
= Industrial/Scientific/Medical - banden
\begin{itemize}
    \item Geen vergunning nodig voor gebruik
    \item Is aan strikte voorwaarden verbonden
    \begin{itemize}
        \item Maximaal vermogen
        \item Maximale transimissietijd
        \item Maximale periodiciteit van transmissies
        \item Maximale bandbreedte
        \item \dots
    \end{itemize}
    \item Specifieke banden zijn hiervoor vrijgegeven
\end{itemize}

\subsection{Interferentie}

\subsubsection{Co-channel interferentie}
Wat als meerdere systemen dezelfde band gebruiken?

\begin{figure}[H]
    \centering
    \includegraphics[width=\textwidth]{Screenshot_20200309_115759.png}
    \caption{Co-channel Interferentie: A en B gebruiken hetzelfde kanaal (6)}
\end{figure}

\bold{Oplossing: Signaalseparatie}

\begin{itemize}
    \item Om een goede signal-to-noise ratio (SNR) te behouden 
    \item Minimaal verschil is afhankelijk van modulatietype
    \item \underline{Voorbeeld:}
    \begin{itemize}
        \item Transmitter A = BPSK $<$ 10dB
        \item Transmitter B = 256-QAM $>$ 40dB
    \end{itemize}
\end{itemize}

\begin{figure}[H]
    \centering
    \includegraphics[width=0.7\textwidth]{Screenshot_20200309_120012.png}
    \caption{Signaalseparatie, A = BPSK, B = 256-QAM}
\end{figure}

\subsubsection{Interferentie van naburige kanalen}
\begin{figure}[H]
    \centering
    \includegraphics[width=0.9\textwidth]{Screenshot_20200309_120313.png}
    \caption{Interferentie van naburige kanalen}
\end{figure}

\subsubsection{Interferentie van andere toestellen}
$\Rightarrow$ Ruis / atmosferische storingen

\begin{figure}[H]
    \centering
    \includegraphics[width=0.9\textwidth]{Screenshot_20200309_120624.png}
    \caption{Interferentie van een microgolfoven}
\end{figure}

\subsection{Meten van interferentie}
\begin{itemize}
    \item Meten van WiFi kanalen kan met software 
    \item Software is niet in staat andere signalen te meten
    \item Kan worden gemeten met een spectrum analyzer
\end{itemize}

\begin{figure}[H]
    \centering
    \includegraphics[width=0.25\textwidth]{Screenshot_20200309_120748.png}
    \caption{Meten van interferentie}
\end{figure}

\subsection{Golflengte}
\begin{itemize}
    \item Golflengte (in meter) is een andere wijze om de frequentie weer te geven
    \item Relevant bij antennes en transmissielijnen
    \item $\lambda = \frac{c}{f}$
    \item $c=$ lichtsnelheid $= 299\ 792\ 458$  m/s
\end{itemize}

\begin{figure}[H]
    \centering
    \includegraphics[width=0.6\textwidth]{Screenshot_20200309_121003.png}
    \caption{Golflengte $\lambda$}
\end{figure}

\subsubsection{Rekenvoorbeeld}
Frequentie van $102.1\text{ MHz}$
\begin{itemize}
    \item $\lambda = \frac{299\ 792\ 458 \text{ m/s}}{102.1\text{ MHz}} = 2.936\dots \text{ m}$
\end{itemize}
"De 70 cm band":
\begin{itemize}
    \item $f = \frac{c}{\lambda} = \frac{299\ 792\ 458 \text{ m/s}}{0.7\ \text{m}} =  428\ 274\ 940\ \text{Hz} = 430\ \text{MHz}$
\end{itemize}

\subsection{Antennesysteem}
\begin{itemize}
    \item Een antenne stuurt / ontvangt zoveel mogelijk van de RF-energie op de gewenste frequentie
    \item $\Rightarrow$ resonantie van het antennesysteem op de gewenste frequentie
    \item Formaat van de antenne is vaak beperkende factor 
\end{itemize}

\begin{figure}[H]
    \centering
    \includegraphics[width=0.3\textwidth]{Screenshot_20200309_122045.png}
    \caption{}
\end{figure}

\subsubsection{De dipool antenne}
\begin{itemize}
    \item Zeer eenvoudig
    \item $\frac12 \lambda$ groot
    \item Vaak toegepast
    \item Kan worden 'opgeplooid'
\end{itemize}

\begin{figure}[H]
    \centering
    \includegraphics[width=0.8\textwidth]{Screenshot_20200309_122306.png}
    \caption{Dipool antenne}
\end{figure}

\subsubsection{Polarisatie van antennes}
\begin{figure}[H]
    \centering
    \includegraphics[width=\textwidth]{Screenshot_20200516_162033.png}
    \caption{Polarisatie van antennes}
\end{figure}

\subsubsection{Directionaliteit van antennes}
\begin{itemize}
    \item Omnidirectionele antenne
    \begin{itemize}
        \item Meestal dipolen of end-fed (=stukje draad)
        \item Discone
    \end{itemize}
    \item Directionele antenne (beam)
    \begin{itemize}
        \item Schotelantenne
        \item Yagi
        \item Patch
    \end{itemize}
\end{itemize}

\begin{figure}[H]
    \centering
    \includegraphics[width=0.5\textwidth]{Screenshot_20200309_122609.png}
    \caption{Antennes: Yagi, Discone, Schotel}
\end{figure}

\subsubsection{Stralingspatroon van antennes}
\begin{itemize}
    \item Gevoeligheid van antennes is niet overal hetzelfde
    \item Zeer afhankelijk van constructie en type antenne
\end{itemize}

\begin{figure}[H]
    \centering
    \includegraphics[width=\textwidth]{Screenshot_20200309_122734.png}
    \caption{Stralingspatroon: kan grafisch geplot worden op meerdere manieren}
\end{figure}

\subsubsection{Gain van een antenne}
\begin{itemize}
    \item Gain = versterking
    \item Uitgedrukt in dB
    \item Nooit 'magisch', er is altijd een trade-off:
    \begin{itemize}
        \item Directionaliteit
        \item Bandbreedte
    \end{itemize}
\end{itemize}

\begin{figure}[H]
    \centering
    \includegraphics[width=0.5\textwidth]{Screenshot_20200309_122914.png}
    \caption{Gain van een antenne}
\end{figure}

\subsection{Propagatie van RF-signalen}
\begin{itemize}
    \item Absorptie
    \item Reflectie
    \item Scattering
    \item Refractie
    \item Path loss
    \item \dots
\end{itemize}

\begin{figure}[H]
    \centering
    \includegraphics[width=0.6\textwidth]{Screenshot_20200309_123000.png}
    \caption{Propagatie van RF-signalen}
\end{figure}

\subsubsection{Path loss}
\begin{itemize}
    \item Verzwakking van het signaal met afstand zonder obstakels $\Rightarrow$ \underline{free space path loss (FSPL)}
    \item \bold{Oorzaak} uitdeinen van het signaal
    \item Verzwakking is exponentieel met de afstand
    \item Enkel afhankelijk van frequentie en afstand
    \item Hogere frequenties hebben hogere path loss
    \item FSPL (dB) $= 20\cdot log_{10}(d) + 20\cdot log_{10}(f) + 32.44$
    \begin{itemize}
        \item d = afstand in km
        \item f = frequentie in MHz
    \end{itemize}
    \item Meer verzwakking bij hogere frequenties
    \item Minder bandbreedte bij lagere frequenties
    \item Wat is beter?
    \begin{itemize}
        \item 2.4GHz WiFi
        \item 5GHz WiFi
    \end{itemize}
\end{itemize}

\begin{figure}[H]
    \centering
    \includegraphics[width=0.5\textwidth]{Screenshot_20200309_123426.png}
    \caption{2.4GHz (minder bandbreedte) vs 5GHz WiFi (meer verzwakking)}
\end{figure}

\subsubsection{Dynamic Rate Shifting (DRS)}
\begin{itemize}
    \item Modulatietechniek dynamisch aanpassen aan de signaalcondities
    \item Minder gunstige SNR $\Rightarrow$ Lagere datarate kiezen
    \item Andere modulatievorm 
\end{itemize}

\begin{figure}[H]
    \centering
    \includegraphics[width=0.8\textwidth]{Screenshot_20200309_123624.png}
    \caption{Dynamic Rate Shifting: modulatievorm verandert bij grotere afstanden}
\end{figure}

\subsubsection{Reflectie}
\begin{itemize}
    \item Signaal wordt gereflecteerd
    \item Bv:
    \begin{itemize}
        \item Door metalen objecten
        \item Water
        \item \dots
    \end{itemize}
\end{itemize}

\begin{figure}[H]
    \centering
    \includegraphics[width=0.4\textwidth]{Screenshot_20200309_124220.png}
    \caption{Reflectie}
\end{figure}

\subsubsection{Absorptie}

\begin{itemize}
    \item Signaal kan deels of volledig worden geabsorbeerd 
    \item Resulteert in verzwakking van bruikbare signaal
    \item Verzwakking = attentuatie typisch in dB
\end{itemize}

\begin{figure}[H]
    \centering
    \includegraphics[width=0.4\textwidth]{Screenshot_20200309_124354.png}
    \caption{Absorptie}
\end{figure}

\subsubsection{Doordringbaarheid}
\begin{itemize}
    \item Hogere frequenties worden gemakkelijk tegengehouden door objecten
    \item Lagere frequenties kunnen gemakkelijker door objecten heen
\end{itemize}

\begin{figure}[H]
    \centering
    \includegraphics[width=0.4\textwidth]{Screenshot_20200309_124451.png}
    \caption{Doordringbaarheid}
\end{figure}

\subsubsection{Scattering}
\begin{itemize}
    \item Signaal wordt gereflecteerd in diverse richtingen
    \item Oneffen oppervlaktes
    \item Wolken materiaal bv zand, \dots
\end{itemize}

\begin{figure}[H]
    \centering
    \includegraphics[width=0.4\textwidth]{Screenshot_20200309_124533.png}
    \caption{Scattering}
\end{figure}

\subsubsection{Refractie}
\begin{itemize}
    \item Bij 2 media met verschillende dichtheid
    \item Afbuiging van het signaal
\end{itemize}

\begin{figure}[H]
    \centering
    \includegraphics[width=0.4\textwidth]{Screenshot_20200309_124610.png}
    \caption{Refractie}
\end{figure}

\subsubsection{Difractie}
\begin{itemize}
    \item RF-signaal wordt beïnvloed door obstakels
    \item Gaat er niet door maar “rond”, zoals water rond een paaltje in een rivier
    \item Verstoort / verminkt het RF-signaal
\end{itemize}

\begin{figure}[H]
    \centering
    \includegraphics[width=0.4\textwidth]{Screenshot_20200309_124659.png}
    \caption{Difractie}
\end{figure}

\subsubsection{Fresnel zones}
\begin{itemize}
    \item Golfmodel RF-signalen 
    \item Worden ook beïnvloed door objecten in nabijheid
    \item Niet enkel door objecten in line-of-sight
\end{itemize}

\begin{figure}[H]
    \centering
    \includegraphics[width=0.4\textwidth]{Screenshot_20200309_124752.png}
    \caption{Fresnel zones}
\end{figure}

\begin{itemize}
    \item Afhankelijk van frequentie en afstand
\end{itemize}

\begin{figure}[H]
    \centering
    \includegraphics[width=0.5\textwidth]{Screenshot_20200309_124825.png}
    \caption{Fresnel zone radius is afhankelijk van frequentie en afstand}
\end{figure}

\begin{itemize}
    \item Communicatie kan verstoord worden, zelfs door een object buiten de line-of-sight
\end{itemize}

\begin{figure}[H]
    \centering
    \includegraphics[width=0.5\textwidth]{Screenshot_20200309_125331.png}
    \caption{Verstoring door object in line-of-sight van transmissie}
\end{figure}

\subsubsection{Propagatie van RF-signalen}

\begin{itemize}
    \item Bepaalde effecten kunnen worden gebruikt
    \item Bv: Communicatie mbv reflectie op de ionosfeer
\end{itemize}

\begin{figure}[H]
    \centering
    \includegraphics[width=0.5\textwidth]{Screenshot_20200309_125500.png}
    \caption{communicatie mbv reflectie op de ionosfeer}
\end{figure}

\subsubsection{MIMO}

\begin{itemize}
    \item = Multiple In / Multiple Out
    \item Reflectie en scattering kan positief worden gebruikt 
    \item Beamforming
    \item Extra processing nodig
\end{itemize}

\begin{figure}[H]
    \centering
    \includegraphics[width=0.5\textwidth]{Screenshot_20200309_125719.png}
    \caption{MIMO}
\end{figure}

\subsection{Multiplexing}
= delen van 1 medium

\begin{itemize}
    \item Verschillende signalen / meerdere deelnemers
    \item Multiplexing $\Rightarrow$ medium $\Rightarrow$ Demultiplexing
\end{itemize}

\subsubsection{FDM - Frequency Division Mux}

\begin{itemize}
    \item Verschillende deelnemers maken gebruik van verschillende frequenties
    \item Zoals radiozenders gelijktijdig uitzenden op hun eigen frequentie
\end{itemize}

\begin{figure}[H]
    \centering
    \includegraphics[width=0.6\textwidth]{fdm.jpg}
    \caption{FDM}
\end{figure}

\subsubsection{WDM - Wavelength Division Mux}
\begin{itemize}
    \item Bij optische signalen
    \item Verschillende golflengte gebruiken over 1 fibre
    \item Vergelijkbaar met FDM
\end{itemize}

\begin{figure}[H]
    \centering
    \includegraphics[width=0.6\textwidth]{wdm.png} 
    \caption{WDM}
\end{figure}

\subsubsection{TDM - Time Division Mux}
\begin{itemize}
    \item Elke transmissie krijgt een tijdslot
    \item Tijdssloten wisselen volgens afspraak 
\end{itemize}

\begin{figure}[H]
    \centering
    \includegraphics[width=0.6\textwidth]{tdm.png} 
    \caption{TDM}
\end{figure}

\subsubsection{OFDM - Orthogonal Frequency Division Mux}
\begin{itemize}
    \item Opsplitsen van bitstream in aparte bitstreams
    \item Deze worden op aparte subcarriers verzonden
    \item Subcarriers overlappen elkaar
    \item Subcarriers zijn in fase gesyncroniseerd
\end{itemize}

\begin{figure}[H]
    \centering
    \includegraphics[width=0.4\textwidth]{ofdm.png} 
    \includegraphics[width=0.4\textwidth]{ofdm2.png} 
    \caption{OFDM}
\end{figure}

\begin{itemize}
    \item Vaak met QAM/QPSK gecombineerd
    \item Uitermate spectraal efficiënt, benadert het theoretisch maximum volgens Nyquist-Shannon
    \item Door de in fase gesynchroniseerde subcarriers hebben de overlappende frequenties geen nadelig effect
\end{itemize}

\begin{figure}[H]
    \centering
    \includegraphics[width=0.5\textwidth]{ofdm3.png} 
    \caption{OFDM: in fase gesynchroniseerd}
\end{figure}

\subsection{Belangrijke eigenschappen voor IoT communicatie}

\begin{itemize}
    \item Kostprijs
    \begin{itemize}
        \item Kost van hardware
        \item Communicatiekost / gebruikskost
        \item Licentiekost
    \end{itemize}
    \item Overbrugbare afstand
    \item Datarate / snelheid
    \item Stroomverbruik
    \item Bereik (beschikbaar, binnen gebouwen)
    \item Standaardisatie
    \item \dots
\end{itemize}

\subsection{Draadloze technologieën}
\begin{itemize}
    \item Ware zoo van beschikbare technologieën
    \item In staat zijn de correcte technologie te kiezen is van zeer groot belang
\end{itemize}

\section{Seriele communicatie}
\subsection{Serieel vs parallel}

\begin{figure}[H]
    \centering
    \includegraphics[width=0.4\textwidth]{Screenshot_20200323_114507.png}
    \caption{Serieel vs parallel}
\end{figure}

\subsection{Parallelle communicatie}
\begin{itemize}
    \item Eenvoudig
    \item Sneller (bij eenzelfde kloksnelheid)
    \item Veel connecties
    \item Clock skew / jitter
\end{itemize}

\begin{figure}[H]
    \centering
    \includegraphics[width=0.5\textwidth]{Screenshot_20200323_114732.png}
    \caption{Parallelle communicatie}
\end{figure}

\subsubsection{Clock skew}
\begin{itemize}
    \item Verschillende vertraging tussen signalen
    \item $\Rightarrow$ Begrensd maximale klokfrequentie
\end{itemize}

\begin{figure}[H]
    \centering
    \includegraphics[width=0.4\textwidth]{Screenshot_20200323_114937.png}
    \caption{Clock skew}
\end{figure}

\subsection{Seriele communicatie}
\begin{itemize}
    \item Minder bekabeling
    \item Minder problemen met timing
    \item Minder last van crosstalk
    \item Complexer / SERDIS
\end{itemize}

\begin{figure}[H]
    \centering
    \includegraphics[width=0.4\textwidth]{Screenshot_20200323_115035.png}
    \caption{}
\end{figure}

\subsection{Serieel vs Parallel: wanneer gebruiken?}
\begin{itemize}
    \item Lange afstanden $\Rightarrow$ serieel, want minder verbindingen
    \item Vroeger: lokale korte verbindingen $\Rightarrow$ parallel
    \item Tegenwoordig: snelle lokale verbindingen $\Rightarrow$ serieel
    \item ATA in harde schijven vervangen door SATA (Serial ATA)
    \item PCI vervangen door PCI Express op moederborden: ook serieel
\end{itemize}

\subsection{SERDIS}
\begin{itemize}
    \item Serialiser
    \item Deserialiser
\end{itemize}

\begin{figure}[H]
    \centering
    \includegraphics[width=0.4\textwidth]{Screenshot_20200323_115743.png}
    \caption{SERDIS}
\end{figure}

\subsubsection{Implementatie}
Met schuifregister
\begin{itemize}
    \item SIPO (serial in parallel out)
    \item PISO (parallel in serial out)
\end{itemize}

\begin{figure}[H]
    \centering
    \includegraphics[width=0.7\textwidth]{Screenshot_20200323_115735.png}
    \caption{Schuifregister}
\end{figure}

\subsection{Duplex}
\begin{itemize}
    \item Full-duplex
    \item Half-duplex
    \item Simplex
\end{itemize}

\begin{figure}[H]
    \centering
    \includegraphics[width=0.6\textwidth]{Screenshot_20200323_115925.png}
    \caption{Verschillende verbindingen}
\end{figure}

\begin{figure}[H]
    \centering
    \includegraphics[width=0.7\textwidth]{Screenshot_20200323_120141.png}
    \caption{Full duplex}
\end{figure}

\subsection{Flow Control}
Beperken hoeveel de zender kan/mag versturen.

\begin{itemize}
    \item Kan in hardware $\Rightarrow$ extra verbindingen
    \item Kan in software $\Rightarrow$ controle karakters
    \item Hardware (CTS / RTS / Enable / \dots)
    \item Xon / Xoff: starten/stoppen van het signaal
\end{itemize}

\subsection{Synchroon vs Asynchroon}
\begin{itemize}
    \item Synchroon $\Rightarrow$ Gebruikt een aparte kloklijn
    \item Asynchroon $\Rightarrow$ Geen kloklijn
\end{itemize}

\begin{figure}[H]
    \centering
    \includegraphics[width=0.7\textwidth]{Screenshot_20200323_120525.png}
    \caption{Kloklijn}
\end{figure}

\subsection{Synchrone seriele communicatie}
\begin{itemize}
    \item 1 device genereert de klok
    \item Alle transmissies synchroniseren op deze klok
    \item Kan op de stijgende flank of op de dalende flank 
\end{itemize}

\begin{figure}[H]
    \centering
    \includegraphics[width=0.4\textwidth]{Screenshot_20200323_120955.png}
    \caption{Synchrone seriele communicatie}
\end{figure}


\subsection{Asynchrone seriele communicatie}
\begin{itemize}
    \item Werkt zonder aparte kloklijn
    \item Dus minder verbindingen
    \item Synchronisatie door ontvanger noodzakelijk
    \item Start en stop bit (/conditie) noodzakelijk
\end{itemize}


\begin{figure}[H]
    \centering
    \includegraphics[width=0.7\textwidth]{Screenshot_20200323_121104.png}
    \caption{Asynchrone seriele communicatie}
\end{figure}

\subsubsection{Parameters}
\begin{itemize}
    \item Baudrate (=bits per seconde)
    \item Aantal databits
    \item Pariteitsbit
    \item Aantal stop bits
\end{itemize}

\subsubsection{Standaarnotering}
\begin{itemize}
    \item Bijvoorbeeld: 9600-8-N-1
    \item Baudrate = 9600
    \item 8 = aantal databits
    \item N = welke pariteit
    \item 1 = aantal stop-bits
\end{itemize}

\subsubsection{Baudrate}
= Snelheid in bits/seconde
\begin{itemize}
    \item Niet alle bits zijn databits
    \item Start, stop en pariteitsbit transporteren geen data
\end{itemize}

\bold{Bijvoorbeeld}
\begin{itemize}
    \item 8-N-1 $\Rightarrow$ 80\% efficientie
    \item 8 databits van in totaal $8+1+1=10\ \text{bits} = 8/10\ \text{of}\ 80\%$
\end{itemize}

\subsubsection{Aantal databits}
\begin{itemize}
    \item Typisch 7 of 8 databits
    \item 7 bit is voldoende voor niet extended ASCII
    \item 8 bit $\Rightarrow$ noodzakelijk voor binaire transmissie (bv voor firmware)
    \item XMODEM / ZMODEM protocollen voor binaire transfers (weinig gebruikt want er bestaan nieuwere protocollen zoals USP)
\end{itemize}

\subsubsection{Pariteit}
= Foutdetectie

\bold{Soorten pariteit}
\begin{itemize}
    \item Even (E)
    \item Odd (O)
    \item None (N)
\end{itemize}

Aantal 1 bits tellen en dit aantal steeds even of oneven maken door een 0 of een 1 toe te voegen.

\subsubsection{Stopbits}
\begin{itemize}
    \item Aantal bits op het einde van een data-bit reeks
    \item $1/1.5/2\ $ bits
\end{itemize}

\subsection{Snelheid}
\begin{itemize}
    \item 1 transmissie = 1 karakter
    \item Aantal bits voor 1 karakter = som alle bits
    \item Baudrate in bits/seconde
\end{itemize}

\bold{Voorbeeldoefening}
\begin{itemize}
    \item 300-8-E-2
    \item 8 databits + 1 pariteit + 2 stopbits + 1 startbit = 12bits
    \item 300 bits/seconde = 300 / 12 = 25 CPS (=characters per second)
\end{itemize}

\subsection{Meerdere deelnemers}
\begin{itemize}
    \item Multi device bus
    \item Daisy chain (=de ene deelnemer hangt aan de andere tot we aan de laatste deelnemer hangen, dan hangt de laatste ook aan de eerste)
    \item $\Rightarrow$ afspraken/arbitrage is noodzakelijk
\end{itemize}


\begin{figure}[H]
    \centering
    \includegraphics[width=0.7\textwidth]{Screenshot_20200323_122755.png}
    \caption{Meerdere deelnemers op een bus}
\end{figure}

\subsection{Manchester encoding}
\begin{itemize}
    \item Manier om asynchroon te werken, maar de klok en datasignaal in 1 signaal te stoppen
    \item Altijd voldoende omschakeling (bv voor RF tranmissie)
    \item Met AND-operatie
\end{itemize}

\begin{figure}[H]
    \centering
    \includegraphics[width=0.9\textwidth]{Screenshot_20200323_123031.png}
    \caption{Manchester encoding: Clock \& Data = OUTPUT}
\end{figure}

\subsection{Differentiele communicatie}
\begin{itemize}
    \item Differential vs single-ended
    \item Zowel synchroon als asynchroon kan differentieel of single-ended werken
    \item Minder gevoelig aan storingen
\end{itemize}


\begin{figure}[H]
    \centering
    \includegraphics[width=0.7\textwidth]{Screenshot_20200323_123305.png}
    \caption{Single-ended (boven) vs Differentieel (onder)}
\end{figure}

\begin{figure}[H]
    \centering
    \includegraphics[width=0.7\textwidth]{Screenshot_20200323_123231.png}
    \caption{Wat er gebeurt bij storing op een differentieel signaal}
\end{figure}

\subsection{Benoeming van signalen}
Deze namen vind je vaak op protocollen en devices zoals de Arduino, RPi, \dots

\begin{itemize}
    \item RX $\Rightarrow$ receiver
    \item TX $\Rightarrow$ transmitter
    \item Dit is mogelijks ambigu. Oplossing: nieuwe naamgeving:
    \begin{itemize}
        \item MOSI $\Rightarrow$ Master Out Slave In
        \item MISO $\Rightarrow$ Master In Slave Out
    \end{itemize}
    \item CS $\Rightarrow$ Chip Select
    \item EN $\Rightarrow$ Enable
    \item R/W $\Rightarrow$ Read/Write
    \item SCL $\Rightarrow$ Serial Clock
    \item SDA $\Rightarrow$ Serial Data
    \item Minder gebruikte signalen (Niet expliciet te kennen)
    \begin{itemize}
        \item CTS 
        \item DTR
    \end{itemize}
    \item \dots
\end{itemize}

\subsection{Protocollen}
= Standaarden voor communicatieafspraken

\begin{figure}[H]
    \centering
    \includegraphics[width=0.7\textwidth]{Screenshot_20200323_124324.png}
    \caption{Timing diagramma dat toont hoe een protocol werkt}
\end{figure}

\section{Bussystemen}

\begin{itemize}
    \item Enkel directe punt-tot-punt verbindingen naar centrale eenheid
    \item Meerdere devices delen een bus
\end{itemize}

$\Rightarrow$ Bussystemen zijn praktischer en (vaak) goedkoper

\subsection{Communicatie op een bus}
\begin{itemize}
    \item Gelijktijdig gebruik van een bus is onmogelijk bij meerdere devices op deze bus
    \item Gebruik van:
    \begin{itemize}
        \item TDM
        \item Tokens (token ring)
        \item CSMA (/CD)
        \item Master / Slave
        \item \dots
    \end{itemize}
\end{itemize}

$\Rightarrow$ Dit zijn allemaal systemen voor arbitrage op een bus

\subsection{Master / Slave communicatie}
\begin{itemize}
    \item Alle communicatie start vanuit de master
    \item Slaves antwoorden op vraag van de master
    \item \underline{Polling} van slaves voor informatie
\end{itemize}

\subsection{Serial Pheripheral Interface (SPI) communicatie}
\begin{itemize}
    \item Synchrone seriele bus (=er is een aparte kloklijn aanwezig)
    \item Full duplex (=communicatie in 2 richtingen op hetzelfde moment mogelijk)
    \item 1 master
    \item 1 of meerdere slaves
\end{itemize}

\subsubsection{Werking}
\begin{itemize}
    \item MOSI (Master Out Slave In)
    \item MISO (Master In Slave Out)
    \item Gemeenschappelijke klok
    \item Chip(/Slave) Select (CS/SS) per slave
\end{itemize}

\begin{figure}[H]
    \centering
    \includegraphics[width=0.7\textwidth]{Screenshot_20200330_113936.png}
    \caption{Werking SPI}
\end{figure}

\subsubsection{Eigenschappen}

\begin{itemize}
    \item Elke lijn heeft slechts 1 driving source (aansturingsbron)
    \item Clock gegenereerd door master
    \item Collisions zijn onmogelijk
    \item Slechts een enkele SPI Slave actief per moment
    \item Actieve Slave door Master gekozen (via Chip Select (CS) lijn)
    \item Een SPI Slave kan geen communicatie aanvragen, behalve via een eventueel aparte lijn (IRQ) los van de SPI bus
\end{itemize}

\subsubsection{Timing Diagram}

\begin{figure}[H]
    \centering
    \includegraphics[width=0.7\textwidth]{Screenshot_20200330_114211.png}
    \caption{SPI Timing Diagram}
\end{figure}

\begin{itemize}
    \item SS laag om communicatie tot stand te brengen
    \item Master genereert een klok na SS
    \item Master stuurt verzoek naar slave: byte wordt doorgeklokt
    \item Slave stuurt op kloksignaal data terug
    \item Master-To-Slave en Slave-To-Master tegelijk is mogelijk want full-duplex
\end{itemize}

\subsection{I2C communicatie}
\begin{itemize}
    \item Inter-Integrated Circuit (IIC $\Rightarrow$ $I^2C$)
    \item Synchrone seriele interface (=aparte kloklijn)
    \item 2 open collector lijnen: Clock \& Data (SCL \& SDA)
    \item Half duplex
    \item 1 of meerdere masters (!) ("multi master bussysteem")
    \item 1 of meerdere slaves
    \item Slaves hebben elk een adres (beslist door fabrikant)
    \item Typisch \underline{lagere communicatiesnelheden} dan SPI
\end{itemize}

\subsubsection{I2C schema}
\begin{itemize}
    \item SCL = Serial Clock
    \item SDA = Serial Data
    \item Beide zijn "Open Collector" $\Rightarrow$ Pull-up weerstanden
\end{itemize}

\begin{figure}[H]
    \centering
    \centerline{
        \includegraphics[width=0.5\textwidth]{Screenshot_20200330_120948.png}
        \includegraphics[width=0.5\textwidth]{Screenshot_20200330_114601.png}
    }
    \caption{I2C schema}
\end{figure}


\subsubsection{I2C Timing Diagram}

\begin{figure}[H]
    \centering
    \includegraphics[width=\textwidth]{Screenshot_20200330_114819.png}
    \caption{I2C Timing Diagram}
\end{figure}

\begin{itemize}
    \item Startconditie
    \item 7 adresbits + 1 R/W bit: master leest (1) of schrijft (0)
    \item Acknowledge-bit
    \item 8 data bits + Acknowledge-bit, herhaaldelijk tot alle data gelezen of gestuurd is
    \item Stopconditie
\end{itemize}

\subsubsection{Eigenschappen}

\begin{itemize}
    \item Slechts 2 lijnen, onafhankelijk van hoeveelheid slaves
    \item Hoger stroomverbruik door pull-up weerstanden
    \item Multi-master mogelijk (complexer, niet vaak gebruikt)
    \item Elke slave heeft een adres (fabriek)
    \item Adresconflicten beperken vaak het gebruik van meerdere dezelfde Slave devices
\end{itemize}

\subsection{Flow Control}
\subsubsection{Problemen}
\begin{itemize}
    \item Kan de ontvanger de data op tijd verwerken?
    \item Is de data klaar om verzonden te worden?
    \item Is er voldoende plaats in een FIFO-buffer (First In First Out, buffer kan overflowen)?
\end{itemize}

\bold{Mogelijke oplossingen:}
\begin{itemize}
    \item Hopen op succes en fouten opvangen (retransmission)
    \item Flow control
\end{itemize}

\subsubsection{Hardware flow control}
\begin{itemize}
    \item RTS/CTS lijnen bij RS232 (wordt gebruikt bij Arduino en RPi)
    \item Clock stretching bij $\text{I}^2\text{C}$: slave maakt de clock trager zodat er meer tijd is om de data te verwerken
    \item \dots
\end{itemize}

\subsubsection{Software flow control}
\begin{itemize}
    \item XON/XOFF: stoppen met zenden (XOFF) tot de data verwerkt is (XON)
    \begin{itemize}
        \item Kan soms voor fouten zorgen
    \end{itemize}
    \item \dots
\end{itemize}

\subsubsection{Geen flow control}
\begin{itemize}
    \item Deterministische timing gebruiken
    \item Retransmisie van ontbrekende data
\end{itemize}

\subsection{Belangrijke bus-interfaces}
\begin{itemize}
    \item UART Logic-Level / RS232
    \begin{itemize}
        \item Variant: RS485 / RS422 (langere afstanden, differentieel)
    \end{itemize} 
    \item SPI
    \item PSP (Parallell Slave Port, dus niet serieel, bij snellere devices gebruikt)
    \item CAN / LIN (zoals $\text{I}^2\text{C}$, maar voor in voertuigen)
    \item $\text{I}^2\text{C}$
    \item $\text{I}^2\text{S}$ (S = sound, voor geluid te transfereren)
    \item 1-wire (Dallas)
\end{itemize}

\section{Foutdetectie en correctie}
\subsection{Integriteitscontrole}
\begin{itemize}
    \item Is de ontvangen data correct?
    \item Data kan 'verminkt' worden
    \begin{itemize}
        \item Door ruis
        \item Door transmissiefouten zoals collisions
        \item Door moedwillige aanpassing (!)
        \item ...
    \end{itemize}
\end{itemize}

\subsection{Detectie van fouten}
\begin{itemize}
    \item Pariteitscontrole
    \item 1 parity bit toevoegen
    \begin{itemize}
        \item Even parity
        \item Odd parity
    \end{itemize}
\end{itemize}

$\Rightarrow$ totale aantal 1-bits even of oneven maken
$\Rightarrow$ afspraak tussen zender en ontvanger moet correct zijn

\subsubsection{Parity check}
\begin{itemize}
    \item Voorbeeld even pariteit: de zender voegt een pariteitsbit toe zodat het aantal 1-bits altijd even is. 
    Als de ontvanger een oneven aantal 1-bits ontvangt, weet hij dat er een fout gebeurd is.
\begin{figure}[H]
    \centering
    \includegraphics[width=\textwidth]{Screenshot_20200420_115619.png}
    \caption{Voorbeeld even pariteit}
\end{figure}
    \item 1-bit fouten worden gedetecteerd (zie character q)
    \item 2-bit fouten worden niet opgemerkt (zie character C)
\end{itemize}

\subsection{Foutcorrectie}
\begin{itemize}
    \item Error correction code (ECC)
    \item Detectie en soms correctie van fouten
    \item \underline{Voorbeelden:}
    \begin{itemize}
        \item Reed-solomon code
        \item Hamming-code 
        \item Turbo code
        \item CRC's
    \end{itemize}
\end{itemize}

\subsubsection{Cyclic Redundancy Check (CRC)}
\begin{itemize}
    \item Kan steeds alle 1-bit fouten detecteren
    \item Kan fouten tot aantal CRC-bits detecteren
    \item Kan sommige andere fouten detecteren, afhankelijk van het gekozen polynoom
\end{itemize}

\begin{figure}[H]
    \centering
    \includegraphics[width=0.6\textwidth]{Screenshot_20200420_121925.png}
    \caption{Voorbeeld CRC}
\end{figure}

\begin{figure}[H]
    \centering
    \includegraphics[width=\textwidth]{Screenshot_20200420_120455.png}
    \caption{Voorbeeld CRC-4 met polynoom $x^3 + 1$}
\end{figure}


\subsection{Integriteitscontrole}
\begin{itemize}
    \item Foutdetectiecodes zijn geen cryptografische functies:
    \begin{itemize}
        \item ECC beschermt tegen toevallige fouten
        \item ECC beschermt niet tegen bewuste manipulatie
    \end{itemize}
    \item Geruik een hash-functie bij integriteitscontrole
    \begin{itemize}
        \item Bijvoorbeeld SHA2 of SHA3
    \end{itemize}
\end{itemize}

\end{document}

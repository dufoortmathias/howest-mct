\documentclass{article}

\usepackage[dutch]{babel}
\usepackage[margin=3cm]{geometry}
\usepackage{graphicx}
\usepackage{float}
\usepackage{caption}
\usepackage{hyperref}
\usepackage{amsmath}
\usepackage{wrapfig}
\usepackage[parfill]{parskip}

% fonts
\usepackage[T1]{fontenc}
\usepackage{helvet}
\renewcommand{\familydefault}{\sfdefault}

\graphicspath{{img/}}
 
\newcommand{\bold}[1]{\textbf{#1}}

%Define the listing package
\usepackage{listings} %code highlighter
\usepackage{upquote}
\usepackage{color} %use color
\definecolor{mygreen}{rgb}{0,0.6,0}
\definecolor{mygray}{rgb}{0.5,0.5,0.5}
\definecolor{mymauve}{rgb}{0.58,0,0.82}

\begin{document}

\begin{titlepage}
    \author{Tuur Vanhoutte}
    \title{Security}
\end{titlepage}

\pagenumbering{gobble}
\maketitle
\newpage
\tableofcontents
\newpage

\pagenumbering{arabic}

\section{Security}

\subsection{Doel}
\begin{itemize}
    \item Security awareness  (bewustwording)
    \item Correcte nomenclatuur (communicatie)
    \item Advies over verantwoordelijkheden
    \item Inzien v/d consequenties v/h falen van security
    \item Situeren en herkennen van problemen
    \item Oplossingen correct implementeren
    \item Correcte methodieken toepassen
\end{itemize}

\subsection{Waarom?}

\begin{itemize}
    \item Niet iedereen heeft even goede bedoelingen
    \item Grote hoeveelheid mensen = veel potenti"ele slachtoffers (internet == iedereen zeer bereikbaar)
    \item Er is geen magische one-size-fits-all oplossing
    \item Verantwoordelijkheid van iedereen
    \item Tegenmaatregelen nemen
    \item Alert en voorzichtig zijn
\end{itemize}


\subsection{Tegenmaatregelen}
\begin{itemize}
    \item Zijn slechts nuttig indien ze effectief worden gebruikt
    \item Lijken vaak in de weg te zitten of lastig, maar zijn noodzakelijk
\end{itemize}

\subsection{Risico}

\begin{figure}[H]
    \centering
    \includegraphics[width=0.4\textwidth]{risk.png}
    \includegraphics[width=0.25\textwidth]{01.png}
    \caption{Risico}
\end{figure}


\begin{itemize}
    \item De mate van bedreiging is niet beheersbaar
    \item De kwetsbaarheid is te reduceren door de implementatie van tegenmaatregelen
    \item Tegenmaatregelen reduceren kwetsbaarheid
    \item Bedrijfsimpact van het risico bepaalt de opportuniteit van de beveiligingsinvestering
    \item Bepalen van de financiële impact van een incident is uitermate bedrijfsspecifiek
\end{itemize}




\subsection{Theoretisch model}
\textcolor{red}{\bold{WORDT GEVRAAGD OP EXAMEN}}

\bold{CIA-model}

\begin{itemize}
    \item Confidentiality (Vertrouwelijkheid)
    \item Integrity (Integriteit)
    \item Availability (Beschikbaarheid)
\end{itemize}

\begin{figure}[H]
    \centering
    \includegraphics[width=0.3\textwidth]{cia-model.png}
    \caption{CIA-model}
\end{figure}

\bold{Vertrouwelijkheid}: gegevens kunnen \textit{enkel} door de juiste partijen worden geraadpleegd.

\bold{Integriteit}: gegevens zijn vaststaand en veranderen niet, tenzij de juiste, gemachtigde personen ze veranderen.

\bold{Beschikbaarheid}: de gegevens zijn beschikbaar en te bekijken door de juiste partijen, ongeacht aanvallen zoals DDOS-attacks.

\url{https://en.wikipedia.org/wiki/Information_security#Confidentiality}

\url{https://en.wikipedia.org/wiki/Information_security#Integrity}

\url{https://en.wikipedia.org/wiki/Information_security#Availability}

\subsubsection{Voorbeelden}

TODO

\subsection{Bedreiging vs kwetsbaarheid}
\bold{Bedreiging (threat)} = potenti"ele negatieve actie dat een ongewenste impact heeft op een computersysteem of applicatie.

\bold{Kwetsbaarheid (vulnerability)} = zwak punt in een computersysteem of applicatie die kan worden ge"exploiteerd. 

\subsubsection{Bedreigde doelen}
\begin{itemize}
    \item Infrastructuur
    \item Gegevens
    \item Operationaliteit
\end{itemize}

\begin{figure}[H]
    \centering
    \includegraphics[width=0.5\textwidth]{crocodiles.jpg}
    \caption{}
\end{figure}

\section{Bedreigingen}

\begin{itemize}
    \item Vallen 1 of meerdere doelen aan
    \item Kunnen toevallig of kwaadwillig beraamd zijn
    \item Gaan uit van `agenten' (personen/organistaties) of gebeurtenissen
\end{itemize}

\subsection{Voorbeelden}

\begin{itemize}
    \item Phishing
    \item Smishing
    \item Vishing
    \item Money rules
    \item Malware
    \item Hardware uit onbetrouwbare bron
    \item Social engineering
    \item \dots
\end{itemize}

\subsection{Types}

\begin{itemize}
    \item Systeemfouten
    \item Gebeurtenissen
    \begin{itemize}
        \item Brand
        \item Stroomuitval
    \end{itemize}
    \item Intern
    \begin{itemize}
        \item Diefstal
        \item Wraak
    \end{itemize}
    \item Extern
    \begin{itemize}
        \item `Hackers'
        \item Spionage
    \end{itemize}
\end{itemize}

\subsection{Phishing}

\begin{itemize}
    \item Oplichting over email
    \item Vaak onwaarschijnlijk verhaal
    \item Vaak herkenbaar malifide links
    \item Soms bijzonder moeilijk herkenbaar
    \item Is de meest voorkomende vorm van fraude
    \item Is de meest uitgebuite kwetsbaarheid van een organisatie
    \item Zo veel mogelijk mensen bereiken, hopen dat een paar mensen toehappen.
\end{itemize}

\subsubsection{Geavanceerde vormen van phishing}

\begin{itemize}
    \item Spear phishing
    \begin{itemize}
        \item doelgerichter
        \item specifiek
        \item afzender spoofen naar iemand die het slachtoffer persoonlijk kent, slachtoffer aanspreken met echte naam
    \end{itemize}
    \item Double barrel attack
    \begin{itemize}
        \item Double barrel = tweeloopsgeweer
        \item Twee emails sturen: 1 heel duidelijk spam, de andere een reactie van de organisatie (bvb bank) die vraagt om op te letten voor phishing mails. 
        \item De tweede mail bevat vaak een link om je wachtwoord te veranderen $\Rightarrow$ link naar valse site
    \end{itemize}
\end{itemize}

\subsubsection{Andere vormen van phishing}

\begin{itemize}
    \item Bank card phishing
    \item CEO-Fraude
    \begin{itemize}
        \item Impersoneren van een CEO om in zijn/haar naam een actie te verrichten
        \item Bvb: leverancier contacteren om betaling op ander rekening nummer te storten
    \end{itemize}
    \item Factuurfraude
    \begin{itemize}
        \item Vroeger: een echte factuur uit een brievenbus nemen, rekeningnummer veranderen en opnieuw in de bus doen
        \item Tegenwoordig: valse facturen opsturen via email
    \end{itemize}
\end{itemize}

\subsubsection{Phishing herkennen}
\begin{itemize}
    \item Afzender controleren
    \item Taalgebruik
    \item Datum controleren: in het weekend moeilijker om om hulp te vragen aan de echte organisatie 
    \item Slachtoffer afschrikken met gerechtelijke stappen ondernemen
    \item Specifieren van extra informatie (bv: u heeft op maandag 01/02/2020 om 16:04 \textit{x} gedaan, daarom moet u nu \textit{y} betalen)
    \item Slachtoffer moet stappen ondernemen om de situatie niet nog erger te maken
    \item Gebruik van legitieme bedrijven om de transactie te voltooien (bv iTuneskaarten, Google Play kaarten, www.becharge.be)
\end{itemize}


\subsection{Smishing}
Oplichting via: \dots
\begin{itemize}
    \item SMS
    \item Whatsapp
    \item Facebook
    \item \dots
\end{itemize}

\subsection{Vishing}
= Voice Sollicitation 

\begin{itemize}
    \item Mensen bellen je op en maken je wijs dat ze u willen helpen om een probleem op te lossen
    \item Vaak pc overnemen met teamviewer en dergelijke
    \item Geld vragen om pc te `herstellen'
    \item Zie ook: refund scams, IRS scams, \dots
\end{itemize}

\subsection{Money mule}
= iemand die zijn/haar bankrekening laat misbruiken voor criminele activiteiten. 

\begin{itemize}
    \item De crimineel contacteert het slachtoffer met een jobaanbieding
    \item De job bestaat uit het overschrijven van bedragen via zijn/haar bankrekening
    \item Voor elke overschrijving 
\end{itemize}

\subsection{Malware}
= Software met als doel kwaad te berokkenen

\begin{itemize}
    \item Trojan
    \item Adware
    \item Virus / worm
    \item Ransomware
    \item Browser Malware
    \item Ook op smartphone
\end{itemize}

\subsection{Ransomware}
Maakt de data op je PC onbruikbaar tot je losgeld betaalt aan de criminelen.

\begin{itemize}
    \item `Kidnappen' van bestanden: bestanden openen niet langer mogelijk
    \item Poging tot innen van losgeld
    \item Vaak via phishing
    \item Enkel een backup van de gegevens kan voldoende beschermen
\end{itemize}

\subsubsection{Voorbeelden}

\begin{itemize}
    \item Wildfire\_locker
    \item Wannacry
    \item Cryptolocker
    \item Bad Rabbit
\end{itemize}

\subsection{Hardware uit onbetrouwbare bron}
\begin{itemize}
    \item USB Rubber ducky
    \begin{itemize}
        \item USB-stick die ergens gedropt wordt (= drop attack), het slachtoffer vindt de USB stick en stopt hem in zijn/haar computer (bvb uit nieuwsgierigheid).
        \item De USB stick werkt als een toetsenbord en typt een attack script op de pc van het slachtoffer
        \item Doel: volledige controle over PC, met bvb remote access (RAT = Remote Access Tool).
    \end{itemize}
\end{itemize}

\subsection{Vreemde netwerken}
\begin{itemize}
    \item Openbare netwerken kunnen worden afgeluisterd
    \item Verkeer op niet-vertrouwde netwerken kan worden omgeleid
\end{itemize}

\begin{figure}[H]
    \centering
    \includegraphics[width=0.5\textwidth]{vreemde-netwerken.png}
    \caption{Vreemde netwerken}
\end{figure}

\subsection{Social engineering}
Een techniek waarbij een crimineel een aanval op computersystemen tracht te ondernemen door de zwakste schakel in de computerbeveiliging, namelijk de mens, te kraken.


\subsection{Bedreigingen: `Agenten'}

\begin{itemize}
    \item Entiteiten waarvan de bedreiging uitgaat
    \item Zijn intern (=werknemers) of extern aan het bedrijf
    \item Kwetsbaarheid voor een agent wordt bepaald door zijn: 
    \begin{itemize}
        \item Toegangsniveau
        \item Kennis
        \item Motivatie
    \end{itemize}
\end{itemize}

\subsubsection{De ontslagen werknemer}
\begin{itemize}
    \item Heeft toegang (nog steeds?) tot de organisatie
    \item Heeft kennis over de werking van de organisatie
    \item Heeft een sterke negatieve motivatie
\end{itemize}

\subsubsection{De `hacker'}
De stereotiepe `hacker':
\begin{itemize}
    \item De blueprint opgevoerd door de media
    \item Is gebaseerd op reele figuren
    \item Vormt een rolmodel voor een bepaalde subcultuur
    \item Het woord `hacker' is vaak nietszeggend
    \item `Script kiddies', `Wannabees', `Crackers'
    \item Bedreiging groot door grote aantallen
    \item Hoofddeksels (hacker ethics):
    \begin{itemize}
        \item Black hat (=informatiecrimineel, voor persoonlijk gewin)
        \item White hat (`for the greater good', `etische hacker')
        \item Gray hat (iets tussen de twee)
    \end{itemize}
\end{itemize}

\bold{De `ethical' hacker}
= iemand die beveiligingen breekt om te tonen dat ze onveilig zijn

\begin{itemize}
    \item Goed of slecht voor security?
    \item Vb: security by obscurity (= niemand weet hoe het werkt dus het is veilig $\Rightarrow$ reeds vele malen slecht idee gebleken)
    \item Penetration testing (= verificatie van beveiliging, maar: mag niet ongevraagd, anders illegaal)
    \item Soms grijze zone
    \item Meldingsplicht? Welke wetgeving? 
    \item Responsible disclosure: firma inlichten ipv volledig internet
\end{itemize}

\subsection{Bedreigingen: gebeurtenissen}

\begin{itemize}
    \item Brand
    \item Stroomuitval
    \item Overstroming
    \item Diefstal
    \item Aanslag
\end{itemize}


\subsection{Threat intelligence}

\begin{itemize}
    \item `Know your enemy'
    \item Noodzakelijk om risico in te schatten
    \item Bijgevolg elementair om te beslissen over opportuniteit van \bold{tegenmaatregelen}
\end{itemize}

Real-time maps

\begin{itemize}
    \item \url{https://www.fireeye.com/cyber-map/threat-map.html}
    \item \url{http://cybermap.kaspersky.com/}
    \item \url{http://map.ipviking.com/}
\end{itemize}

\begin{figure}[H]
    \centering
    \includegraphics[width=0.8\textwidth]{threat-intelligence.png}
    \caption{Threat intelligence}
\end{figure}

\section{Beveiligen}

\subsection{Herhaling: kwetsbaarheden}

\begin{itemize}
    \item Software vulnerabilities
    \begin{itemize}
        \item Geen updates
        \item Foutief patch management
    \end{itemize}
    \item Interne toegang
    \begin{itemize}
        \item Misbruik machtigingen
        \item Wraak / ontslaan van werknemer
    \end{itemize}
    \item Extern bereikbare diensten
    \item Phishing / spear phishing
    \begin{itemize}
        \item The human factor
        \item Meest gebruikte entrypoint
        \item Email (SMTP) is niet geauthentiseerd
    \end{itemize}
\end{itemize}

\subsection{Shodan search engine demo}

\begin{itemize}
    \item \url{http://www.shodanhq.com}
    \item Zoekt naar geconnecteerde devices
    \item Webcams, videofoons, windturbines, waterkrachtcentrales, PLC's, \dots
\end{itemize}


\subsection{ICT security}
\begin{itemize}
    \item Is zeer complex
    \item Omvat erg veel, zeer diverse kennisdomeinen
    \item Wordt erg vaak over-gesimplificeerd
\end{itemize}

\subsubsection{Usability vs Security}
\bold{Extremen:}

\begin{itemize}
    \item Totale security is enkel mogelijk bij onbestaande usability
    \item Optimale usability is enkel mogelijk bij onbestaande security
\end{itemize}

In elke security implementatie zijn deze 3 factoren nodig:

\begin{enumerate}
    \item Security 
    \item Functionality
    \item Ease of Use
\end{enumerate}

We moeten zoeken naar een gebalanceerde compromis voor alle stakeholders.

Een bruikbare infrastructuur kan nooit 100\% veilig zijn $\Rightarrow$ voorzichtig afwegen van alle parameters en belangen.

\bold{Voorbeeld: } een fingerprint reader: handig, maar niet zo veilig. Iemand met slechte bedoelingen kan de vingerafdruk kopi"eren.

\subsection{Tegenmaatregelen (mitigation)}

\begin{itemize}
    \item Corporate policy  - Training  - Awareness
    \item Coding practices
    \item Testing (Pentesting)
    \item Vulnerability management
    \item Backup
    \item Disaster recovery plan
    \item Fysieke Security
    \item Firewalls / IDS / IPS
\end{itemize}

\subsubsection{Defense in depth strategie}
\textcolor{red}{\bold{WORDT GEVRAAGD OP EXAMEN}}

\begin{itemize}
    \item Layered security
    \item Strategie bij incident
    \item Plannen en documenteren
    \item Nooit alle eieren in 1 mandje leggen
\end{itemize}

\begin{figure}[H]
    \centering
    \includegraphics[width=0.5\textwidth]{layered-security.png}
    \caption{Layered security}
\end{figure}

\subsection{ICC / Belgian Cyber Security Guide}

\begin{itemize}
    \item Checklist
    \item Do's \& Dont's
    \item Gratis te downloaden: \url{http://iccbelgium.be/becybersecure/}
\end{itemize}


\section{Beveiligen van toegang}

\subsection{Authorisatie vs authenticatie}

\begin{itemize}
    \item Authorisatie = een gebruiker kan bepaalde dingen wel of niet doen, afhankelijk van zijn/haar beveiligingsniveau
    \item Authenticatie = is de gebruiker wel wie hij/zij beweert te zijn? Controleren met 1 of meer beveiligingsbasissen.
\end{itemize}

\subsection{Beveiligingsbasissen}
\textcolor{red}{\bold{WORDT GEVRAAGD OP EXAMEN}}

3 opties: 
\begin{itemize}
    \item Weten
    \item Hebben
    \item Zijn
\end{itemize}

\subsection{Beveiliging op 'Weten'-basis: wachtwoorden}

\begin{itemize}
    \item Meest gebruikte bron van authenticatie
    \item Moet voldoende sterk zijn
    \begin{itemize}
        \item Lengte (minstens 12 tekens)
        \item Verschillende soorten tekens
        \item Geen bestaande woorden of logische sequenties
    \end{itemize}
\end{itemize}

\subsubsection{Entropie}

= `wanorde'

= hoeveel mogelijkheden er zijn $\Rightarrow$ hoe sterk een wachtwoord is

\begin{figure}[H]
    \centering
    \includegraphics[width=0.5\textwidth]{wachtwoord-entropie.png}
    \includegraphics[width=0.3\textwidth]{wachtwoord-entropie2.png}
    \caption{Entropie van een wachtwoord}
\end{figure}

\begin{itemize}
    \item Entropie = uitgedrukt in bits
    \item Beste wachtwoorden zijn vooral voldoende lang
    \item Opgelet voor wachtwoorden in woordenlijsten
    \item Vaak gebruikte wachtwoorden zijn gekend
    \item Enorme lijsten met wachtwoorden zijn beschikbaar
    \item Vaak succesvolle aanval op anders toch complexe wachtwoorden
    \item \url{https://howsecureismypassword.net/}
    \item \url{https://haveibeenpwned.com/}
\end{itemize}

\subsubsection{Tips}

\begin{itemize}
    \item Gebruik geen logisch patroon
    \item Mijd hergebruik voor verschilende diensten (ook niet azertyTwitter en azertyFacebook, etc)
    \item Wijzig je wachtwoorden regelmatig
    \item Leen nooit een wachtwoord uit aan iemand anders
    \item Gebruik waar mogelijk een $\text{2}^{\text{de}}$ factor voor authenticatie (2FA) of multi-factor authenticatie (MFA)
    \item Maak eventueel gebruik van een \underline{wachtwoordkluis}
    \item Let erop door de organisatie goedgekeurde wachtwoordkluis-software te gebruiken
    \item Noteer wachtwoorden NOOIT waar deze door derden kunnen worden achterhaald
\end{itemize}

\subsection{Beveiliging op `Hebben'-basis}

\begin{itemize}
    \item Smartcards
    \item Dongles
    \item Transponder (=soort sleutel)
    \item Digipass (=merk van authenticatiediensten en -producten)
    \item Google Authenticator (=smartphone-app)
\end{itemize}

\subsection{Beveiliging op `Zijn'-basis': biometrische beveiliging}

\begin{itemize}
    \item Iris-scanner
    \item Vingerafdruk
    \item Stem
\end{itemize}

\subsection{Combinatie van meerdere authenticatiemethodes}

\begin{itemize}
    \item 2FA en MFA
    \item Bij voorkeur kiezen tussen methodes op \textit{verschillende} werkingsbasis
\end{itemize}

\subsection{Fysische toegang}

\begin{itemize}
    \item Onvergrendelde schermen
    \item Toegangscontrole serverroom
    \item Hardware aanpassingen of diefstal
    \begin{itemize}
        \item Asset management software
    \end{itemize}
    \item Toegang tot het netwerk
    \item Introductie van vreemde software
    \item Opstarten vanaf andere media
\end{itemize}

\subsection{Privilege escalation}

= zichzelf op een ander gebruikersniveau zetten

\begin{itemize}
    \item Horizontal escalation
    \begin{itemize}
        \item Session hijacking van een andere gebruiker
        \item = toegang verkrijgen van het account van een andere gebruiker
        \item bv: inloggen in Facebook en via je account in het account van een andere gebruiker geraken
    \end{itemize}
    \item Vertical escalation
    \begin{itemize}
        \item = `privilege elevation'
        \item Meer machtigingen verwerven
    \end{itemize}
\end{itemize}


\section{Backup}

\begin{itemize}
    \item Essentieel voor het beschermen van data
    \item Concrete back-up policy
    \item Disaster recovery paln
\end{itemize}

\subsection{Veelgebruikte backup-media}
\begin{itemize}
    \item Tape
    \item Harddisk
    \item USB-stick
    \item Cloud-backup
\end{itemize}

\bold{RAID is geen backup:} beschermt niet tegen meeste risico's. Helpt alleen bij falen van de harddisk. 

\subsection{LTO Tapes}

LTO = Linear Tape Open = open standaard

\begin{itemize}
    \item Worden nog altijd gebruikt, up-to-date
    \item Hoge capaciteit (10-12TB per tape)
    \item Hoge transfer rate (500MB/s)
    \item Redeljk goedkoop, iets duurder dan harde schijven per GB
\end{itemize}

\subsubsection{LTO Drive}


\begin{itemize}
    \item Toestel om LTO Tapes te lezen/schrijven
    \item Heel duur (duizenden euros)
\end{itemize}

\begin{figure}[H]
    \centering
    \includegraphics[width=0.2\textwidth]{lto-drive-single.png}
    \caption{LTO drive voor 1 tape}
\end{figure}

\begin{figure}[H]
    \centering
    \includegraphics[width=0.3\textwidth]{lto-drive-multi.png}
    \caption{LTO tape robot: 12000-15000 euro}
\end{figure}


\subsection{Eigenschappen van een correcte backup}

\begin{itemize}
    \item Offline
    \item Beveiling tegen aanpassing (Integrity)
    \item Beschikbaar (Availability)
    \item Veilig opgeslegen (Confidentiality)
    \item Betrouwbaar 
\end{itemize}

\subsection{3-2-1 regel}

\begin{figure}[H]
    \centering
    \includegraphics[width=0.3\textwidth]{321-regel.png}
    \caption{3-2-1-regel: 3 copies, 2 formats, 1 offsite}
\end{figure}


\subsection{Cloud back-up}

\begin{itemize}
    \item Wat met aansprakelijkheid?
    \begin{itemize}
        \item Je hebt geen controle over het systeem
    \end{itemize}
    \item Wat met beschikbaarheid?
    \begin{itemize}
        \item Niet zo makkelijk om de hele backup te downloaden uit de cloud (bandbreedtelimieten, snelheid, \dots)
    \end{itemize}
    \item Wat met vertrouwelijkheid?
    \begin{itemize}
        \item Wie heeft allemaal toegang tot de data?
    \end{itemize}
    \item Wetgeving?
    \item Wel zeer eenvoudig en gemakkelijk
\end{itemize}

\subsection{Back-up policy}

\begin{itemize}
    \item Hoge back-up frequentie
    \item Goede back-up strategie
    \item Type back-up
    \item Opslag van back-up:
    \begin{itemize}
        \item Dicht bij de server voor snelle toegang
        \item Op een andere locatie voor veiligheid
    \end{itemize}
    \item Controle van integriteit back-up
    \item Testen van disaster-recovery plan
\end{itemize}

\subsubsection{Grootvader - vader - zoon-systeem}

\textcolor{red}{\bold{WORDT GEVRAAGD OP EXAMEN}}

\begin{figure}[H]
    \centering
    \includegraphics[width=0.5\textwidth]{gfs.png}
    \caption{Grootvader-vader-zoon systeem}
\end{figure}

\bold{Bv:} Elke maandag een backup op de `Father'-schijf, elke andere weekdag een backup op de `Zoon'-schijf, elke maand een backup op de `Grootvader'-schijf.

\begin{itemize}
    \item Back-uprotatie:
    \begin{itemize}
        \item Meerdere backups, waarbij de oudste backup wordt overschreden bij het maken van een nieuwe backup.
        \item Zo heb je altijd een chronologische opeenvolging van backups
    \end{itemize}
    \item Archivering
    \item Backup moet zowel:
    \begin{itemize}
        \item Actueel zijn
        \item Voldoende teruggaan in de tijd
    \end{itemize}
\end{itemize}

\url{https://en.wikipedia.org/wiki/Backup_rotation_scheme#Grandfather-father-son}

\subsection{Belangrijk}

\begin{itemize}
    \item CIA-model over de hele lijn
    \item Moet worden getest
    \item Moet effectief worden uitgevoerd
\end{itemize}

\end{document}
\documentclass{article}

\usepackage[dutch]{babel}
\usepackage[margin=3cm]{geometry}
\usepackage{graphicx}
\usepackage{float}
\usepackage{caption}
\usepackage{hyperref}
\usepackage{amsmath}
\usepackage{wrapfig}
\usepackage[parfill]{parskip}

% fonts
\usepackage[T1]{fontenc}
\usepackage{helvet}
\renewcommand{\familydefault}{\sfdefault}

\graphicspath{{img/}}
 
\newcommand{\bold}[1]{\textbf{#1}}

%Define the listing package
\usepackage{listings} %code highlighter
\usepackage{upquote}
\usepackage{color} %use color
\definecolor{mygreen}{rgb}{0,0.6,0}
\definecolor{mygray}{rgb}{0.5,0.5,0.5}
\definecolor{mymauve}{rgb}{0.58,0,0.82}
 
%Customize a bit the look
\lstset{ %
backgroundcolor=\color{white}, % choose the background color; you must add \usepackage{color} or \usepackage{xcolor}
basicstyle=\footnotesize, % the size of the fonts that are used for the code
breakatwhitespace=false, % sets if automatic breaks should only happen at whitespace
breaklines=true, % sets automatic line breaking
captionpos=b, % sets the caption-position to bottom
commentstyle=\color{mygreen}, % comment style
deletekeywords={...}, % if you want to delete keywords from the given language
escapeinside={\%*}{*)}, % if you want to add LaTeX within your code
extendedchars=true, % lets you use non-ASCII characters; for 8-bits encodings only, does not work with UTF-8
frame=single, % adds a frame around the code
keepspaces=true, % keeps spaces in text, useful for keeping indentation of code (possibly needs columns=flexible)
keywordstyle=\color{blue}, % keyword style
% language=Octave, % the language of the code
morekeywords={*,...}, % if you want to add more keywords to the set
numbers=left, % where to put the line-numbers; possible values are (none, left, right)
numbersep=5pt, % how far the line-numbers are from the code
numberstyle=\tiny\color{mygray}, % the style that is used for the line-numbers
rulecolor=\color{black}, % if not set, the frame-color may be changed on line-breaks within not-black text (e.g. comments (green here))
showspaces=false, % show spaces everywhere adding particular underscores; it overrides 'showstringspaces'
showstringspaces=false, % underline spaces within strings only
showtabs=false, % show tabs within strings adding particular underscores
stepnumber=1, % the step between two line-numbers. If it's 1, each line will be numbered
stringstyle=\color{mymauve}, % string literal style
tabsize=2, % sets default tabsize to 2 spaces
title=\lstname % show the filename of files included with \lstinputlisting; also try caption instead of title
}
%END of listing package%

\lstdefinelanguage{CSS}{
      keywords={accelerator,azimuth,background,background-attachment,
            background-color,background-image,background-position,
            background-position-x,background-position-y,background-repeat,
            behavior,border,border-bottom,border-bottom-color,
            border-bottom-style,border-bottom-width,border-collapse,
            border-color,border-left,border-left-color,border-left-style,
            border-left-width,border-right,border-right-color,
            border-right-style,border-right-width,border-spacing,
            border-style,border-top,border-top-color,border-top-style,
            border-top-width,border-width,bottom,caption-side,clear,
            clip,color,content,counter-increment,counter-reset,cue,
            cue-after,cue-before,cursor,direction,display,elevation,
            empty-cells,filter,float,font,font-family,font-size,
            font-size-adjust,font-stretch,font-style,font-variant,
            font-weight,height,ime-mode,include-source,
            layer-background-color,layer-background-image,layout-flow,
            layout-grid,layout-grid-char,layout-grid-char-spacing,
            layout-grid-line,layout-grid-mode,layout-grid-type,left,
            letter-spacing,line-break,line-height,list-style,
            list-style-image,list-style-position,list-style-type,margin,
            margin-bottom,margin-left,margin-right,margin-top,
            marker-offset,marks,max-height,max-width,min-height,
            min-width,-moz-binding,-moz-border-radius,
            -moz-border-radius-topleft,-moz-border-radius-topright,
            -moz-border-radius-bottomright,-moz-border-radius-bottomleft,
            -moz-border-top-colors,-moz-border-right-colors,
            -moz-border-bottom-colors,-moz-border-left-colors,-moz-opacity,
            -moz-outline,-moz-outline-color,-moz-outline-style,
            -moz-outline-width,-moz-user-focus,-moz-user-input,
            -moz-user-modify,-moz-user-select,orphans,outline,
            outline-color,outline-style,outline-width,overflow,
            overflow-X,overflow-Y,padding,padding-bottom,padding-left,
            padding-right,padding-top,page,page-break-after,
            page-break-before,page-break-inside,pause,pause-after,
            pause-before,pitch,pitch-range,play-during,position,quotes,
            -replace,richness,right,ruby-align,ruby-overhang,
            ruby-position,-set-link-source,size,speak,speak-header,
            speak-numeral,speak-punctuation,speech-rate,stress,
            scrollbar-arrow-color,scrollbar-base-color,
            scrollbar-dark-shadow-color,scrollbar-face-color,
            scrollbar-highlight-color,scrollbar-shadow-color,
            scrollbar-3d-light-color,scrollbar-track-color,table-layout,
            text-align,text-align-last,text-decoration,text-indent,
            text-justify,text-overflow,text-shadow,text-transform,
            text-autospace,text-kashida-space,text-underline-position,top,
            unicode-bidi,-use-link-source,vertical-align,visibility,
            voice-family,volume,white-space,widows,width,word-break,
            word-spacing,word-wrap,writing-mode,z-index,zoom},  
      sensitive=true,
      morecomment=[l]{//},
      morecomment=[s]{/*}{*/},
      morestring=[b]',
      morestring=[b]",
      alsoletter={:},
      alsodigit={-}
    }

\begin{document}

\begin{titlepage}
    \author{Tuur Vanhoutte}
    \title{Interaction design}
\end{titlepage}

\pagenumbering{gobble}
\maketitle
\newpage
\tableofcontents
\newpage

\pagenumbering{arabic}

\section{CSS Variables}

\subsection{Wat?}

\begin{itemize}
    \item CSS custom properties for cascading variables.
    \item CSS Variables = Custom Properties.
    \item Relatief nieuwe manier om veelgebruikte values om te zetten naar Variables.
    \item To keep consistency, set global variables for everything except layout values.
\end{itemize}

\subsection{Opbouw \& Syntax}

\subsubsection{Custom property: defining the variable}
Custom property: value

\begin{lstlisting}[language=CSS]
--my-cool-color: HotPink;
\end{lstlisting}[language=CSS]

\subsubsection{Cascading variable: applying the variable}

Applying your custom property using the var() function

\begin{lstlisting}[language=CSS]
var(--my-cool-color)
\end{lstlisting}[language=CSS]

\subsection{Syntax}

\begin{lstlisting}[language=CSS]
:root {
    --my-cool-color: HotPink;
}

p {
    color: var(--my-cool-color);
}

.foo {
    background-color: var(--my-cool-color);
}
\end{lstlisting}[language=CSS]

\subsubsection{CSS Variables Are Case Sensitive}

\begin{lstlisting}[language=CSS]
:root {
    --foo: HotPink;
    --FOO: #BADA55;
}
p {
    color: var(--foo);
}
.foo {
    background-color: var(--FOO);
}
\end{lstlisting}[language=CSS]


\subsubsection{This is wrong}

\begin{lstlisting}[language=CSS]
.test {
    --side: margin-top;
    var(--side): 20px
}
\end{lstlisting}[language=CSS]

\subsubsection{Use the calc() function to do math}

\begin{lstlisting}[language=CSS]
:root {
    --whitespace: 20px;
    --whitespace-lg: var(--whitespace) * 2; /* won't work */
    --whitespace-lg: calc(var(--whitespace) * 2); /* correct */
}
\end{lstlisting}[language=CSS]

\subsubsection{Kan ook shorthand values bevatten}

\begin{lstlisting}[language=CSS]
:root {
    --transition: all .1s ease-out;
}
a {
    transition: var(--transition);
}   
\end{lstlisting}[language=CSS]

\subsubsection{Kan ook bestaan uit andere variables.}

\begin{lstlisting}[language=CSS]
:root {
    --transition-property: all;
    --transition-duration: .1s;
    --transition-timing-function: ease-out;
    --transition: var(--transtion-property) 
        var(--transition-duration) 
        var(--transition-timing-function);
}

a {
    transition: var(--transition);
}
\end{lstlisting}[language=CSS]

\subsubsection{Default values}

\begin{lstlisting}[language=CSS]
.c-button {
    border: 1px solid var(--button-color, HotPink);
    background-color: transparent;
}
.c-button:hover {
    background-color: var(--button-color, HotPink);
}
.c-button--beta {
    --button-color: #BADA55;
}
\end{lstlisting}[language=CSS]

\subsubsection{Default values with other variables}

\begin{lstlisting}[language=CSS]
:root {
    --color-pink: HotPink;
    --color-badass: #BADA55;
}
.c-button {
    border: 1px solid var(--button-color, var(--color-pink));
}
.c-button:hover {
    background-color: var(--button-color, var(--color-pink));
}
.c-button--beta {
    --button-color: var(--color-badass);
\end{lstlisting}[language=CSS]

\subsection{Cascade}
CSS Variables Are Subject to the Cascade and Inheritance rules

\url{https://codepen.io/simoncoudeville-nmct/pen/BaBMGPZ}

\begin{lstlisting}[language=CSS]
:root {
    --my-cool-color: HotPink;
}
/* Alle elementen binnen de root waar je --
my-cool-color variable toepast zullen de
value HotPink overerven. */

p {
color: var(--my-cool-color);
}

.foo {
/* Elke paragraaf in het element met
de class ".foo" krijgt de kleur #BADA55.*/

--my-cool-color: #BADA55;
}
\end{lstlisting}[language=CSS]

\url{https://codepen.io/simoncoudeville-nmct/pen/aboMBop}

\subsubsection{CSS variables can be made conditional with @media and other conditional rules}

\begin{lstlisting}[language=CSS]
:root {
    --whitespace: 1em;
}
@media screen and (min-width: 768px) {
    :root {
        --whitespace: 2em;
    }
}
.c-card {
    padding: var(--whitespace);
}

\end{lstlisting}[language=CSS]


\url{https://codepen.io/simoncoudeville-nmct/pen/JjXaQwB}

\subsubsection{Ideal for dark themes}

\begin{lstlisting}[language=CSS]
:root {
    --color: white;
    --background-color: black
}
@media (prefers-color-scheme: dark) {
    :root {
    --color: black;
    --background-color: white
    }
}
.html {
    background-color: var(--background-color);
    color: var(--color);
}
\end{lstlisting}[language=CSS]

\url{https://codepen.io/simoncoudeville-nmct/pen/eYOxbPp}

\subsection{Hoisting}
= Accessing a Variable First and Declaring Later

\begin{lstlisting}[language=CSS]
body{
    background: var(--bg-fill);
}
:root{
    --bg-fill: green;
}

\end{lstlisting}[language=CSS]

\subsection{Scoped variables}

Twee soorten:
\begin{itemize}
    \item Global Scoped Variables
    \item Local Scoped Variables
\end{itemize}

\subsubsection{Global scoped variables}

\begin{lstlisting}[language=CSS]
:root {
    --global-color: black;
}
\end{lstlisting}[language=CSS]
\begin{itemize}
    \item :root is a CSS pseudo-class selector used to select the element that represents the root of the document.
    \item :root is hetzelfde als html maar is specifieker
    \item :root = html
    \item :root > html
\end{itemize}

\url{https://codepen.io/simoncoudeville-nmct/pen/OJLBKXq}

Global variables kan je overal hergebruiken en overschrijven. Ideaal dus voor values die veel hergebruikt worden:

\begin{itemize}
    \item colors
    \item whitespace
    \item border-radius
    \item transitions
    \item \dots
\end{itemize}

\subsubsection{Local scoped variables}
\begin{itemize}
    \item Local scoped variables worden gedeclareerd binnen een specifieke selector.
    \item Local scoped variables hebben access tot global scoped variables.
    \item Ideaal voor components.
\end{itemize}

\begin{lstlisting}[language=CSS]
.alert {
    --alert-color: #222;
    color: var(--alert-color);
    border-color: var(--alert-color);
}
:root {
    --global-fontSize: 16px;
}
.c-button {
    --button-fontSize: var(--global-fontSize);
    font-size: var(--button-fontSize);
}
\end{lstlisting}[language=CSS]


\subsection{Naming system}

\subsubsection{The two-level theming system}

Systeem om global variables te gebruiken in local variables.

\subsubsection{Global level}

De hoofdreden om global variables te hebben is consistentie.

They are prefixed with the word global and follow this formula:

-{}-global-{}-concept-{}-modifier-{}-state-{}-propertyCamelCase

\begin{itemize}
    \item a concept is something like a spacer, main-title or text
    \item a state is something like hover, or expanded
    \item a modifier is something like sm, or lg
    \item and a propertyCamelCase is something like backgroundColor or fontSize
\end{itemize}

\subsubsection{Local level}

They follow this formula: 

-{}-block\_\_element-{}-modifier-{}-state-{}-propertyCamelCase

\begin{itemize}
    \item The block\_\_element-{}-modifier selector name is something like alert\_\_actions or alert--primary
    \item a state is something like hover or active
    \item The value of component scoped variables is always defined by a global variable.
\end{itemize}


\begin{lstlisting}[language=CSS]
.c-alert {
    /* Component scoped variables are always defined by global variables */
    --c-alert--Padding: var(--global--spacer--md);
    --c-alert--primary--BackgroundColor: var(--global--primary-color);
    --c-alert__title--FontSize: var(--global--secondary-title--fontSize);
    /* --block--propertyCamelCase */
    padding: var(--c-alert--padding);
}

/* --block--state--propertyCamelCase */
.c-alert--primary {
    background-color: var(--c-alert--primary--backgroundColor);
}

/* --block__element--propertyCamelCase */
.c-alert__title {
    font-size: var(--c-alert__title--fontSize);
}
\end{lstlisting}[language=CSS]

\subsection{Herhalingsvragen}

\begin{itemize}
    \item Hoe worden CSS variables nog genoemd?
    \item Hoe worden CSS variables opgebouwd?
    \item Wat is CSS Hoisting?
    \item Wat is het verschil tussen global en local variables?
\end{itemize}

\url{https://codepen.io/simoncoudeville-nmct/pen/vYBbbXz?editors=1100}

\section{Forms}

\subsection{Form}
\begin{itemize}
    \item HTML forms are a very powerful tool for interacting with users
    \item Groot deel van een digitale interface
    \item In veel vormen en maten
\end{itemize}

\subsubsection{Voorbeelden}
\begin{itemize}
    \item Nieuwe repo maken op github
    \item Profiel aanmaken
    \item Route kiezen DeLijn
    \item \dots
\end{itemize}

\subsubsection{Basic form syntax}

\begin{lstlisting}[language=HTML]
<form action="">
    <input type="">
    ...
    Submit
</form>
\end{lstlisting}

\bold{Maak geen forms zonder (submit) button!} Zonder (submit) button kan je niet op enter duwen

\subsection{Input types}

\begin{itemize}
    \item Alle input types: \url{https://www.w3schools.com/tags/att_input_type.asp}
    \item Dit zijn de belangrijkste types die je moet kennen:
    \begin{itemize}
        \item Text-achtigen
        \item Time-achtigen
        \item Option-achtigen
        \item Textarea
        \item Select
        \item Range
        \item Hors catégorie
    \end{itemize} 
\end{itemize}

\subsubsection{HTML5 input types}

\begin{itemize}
    \item Browser validatie
    \item Smartphone keyboard verandert
    \item Testen op je smartphone: 
    
    \url{https://mobiforge.com/design-development/html5-mobile-
    web-forms-and-input-types}
    \item Live demo: \url{https://codepen.io/simoncoudeville-nmct/pen/gQYqBY}
\end{itemize}

\subsubsection{Text-achtigen}

\begin{itemize}
    \item Zien er visueel ongeveer hetzelfde uit
    \item Op smartphones: keyboard veranderd op basis van welk type de input is
    \item \bold{text} - basis input type voor HTML5 input types
    \item \bold{password} - vervangt de letters door bolletjes
    \item \bold{email} - valideert de browser enkel als je een correct e-mail adres invult
    \item \bold{number}
    \begin{itemize}
        \item Aanvaardt enkel nummers
        \item Heeft pijltjes om te vermeerderen of te verminderen
    \end{itemize}
    \item \bold{tel} - telefoonnummers
\end{itemize}

\subsubsection{Time-achtigen}

\begin{itemize}
    \item Zien er visueel ongeveer hetzelfde uit als de text-achtigen
    \item date - toont een native date picker
    \item week - toont een variant van de native date picker
    \item month - toont een variant van de native date picker
    \begin{itemize}
        \item Vooral date zal je kunnen gebruiken.
    \end{itemize}
    \item number - aanvaardt enkel nummers
    \item tel - handig voor mobile
\end{itemize}

\begin{figure}[H]
    \centering
    \includegraphics[width=0.5\textwidth]{input-date-time.png}
    \caption{<input type="date"> en <input type="time"> op smartphones}
\end{figure}

\subsubsection{Option-achtigen}

\bold{checkbox}

\begin{itemize}
    \item Meerdere keuzes mogelijk uit een aantal keuzes
    \item Of kan alleen bestaan
\end{itemize}

\bold{radio(button)}

\begin{itemize}
    \item Slechts 1 keuze mogelijk uit een aantal keuzes
    \item Kan niet alleen bestaan, altijd in en group
    \item Gekoppeld aan elkaar door het name attribuut
    \item Niet voor enkele binaire keuzes
    \item Kan je ook customizen met CSS
\end{itemize}

Geschiedenis van de radio-button: \url{https://www.jitbit.com/radio-button/}

\begin{figure}[H]
    \centering
    \includegraphics[width=0.4\textwidth]{radiobuttons.png}
    \includegraphics[width=0.3\textwidth]{checkboxes.png}
    \caption{}
\end{figure}

\bold{Laat een checkbox er nooit uitzien als een radio button en omgekeerd!}

\subsection{Checkbox or toggle?}
\begin{figure}[H]
    \centering
    \includegraphics[width=0.2\textwidth]{toggle-switch.png}
    \caption{Links: toggle switch, rechts: checkbox}
\end{figure}

\subsubsection{Checkbox}
\begin{itemize}
    \item Checked of niet
    \item Heeft nog confirmatie nodig
    \item Meerdere opties die bij elkaar horen
    \item Checken van sub options (intermediate state)
    \item Enkele ja/nee optie
\end{itemize}

\subsubsection{Toggle switch = veredelde checkbox}
\begin{itemize}
    \item Aan of af zetten
    \item Instant response zonder confirmatie
    \item Afzonderlijke features of settings
    \item Enkele aan/af beslissing
\end{itemize}

\subsubsection{Voorbeelden}

\begin{figure}[H]
    \centering
    \includegraphics[width=0.5\textwidth]{toggle-ex-1.png}
    \caption{De opties die een directe reactie vereisen kunnen het best geselecteerd worden met een toggle switch.}
\end{figure}

\begin{figure}[H]
    \centering
    \includegraphics[width=0.5\textwidth]{toggle-ex-2.png}
    \caption{Checkboxes hebben de voorkeur wanneer een expliciete actie vereist is om instellingen toe te passen.}
\end{figure}

\begin{figure}[H]
    \centering
    \includegraphics[width=0.5\textwidth]{toggle-ex-3.png}
    \caption{Selecteren van meerdere opties in een lijst biedt betere ervaring met checkboxes.}
\end{figure}

\begin{figure}[H]
    \centering
    \includegraphics[width=0.5\textwidth]{toggle-ex-4.png}
    \caption{Indeterminate state wordt het best voorgesteld met een checkbox. (zie `All' aan de linkerkant)}
\end{figure}

\begin{figure}[H]
    \centering
    \includegraphics[width=0.5\textwidth]{toggle-ex-5.png}
    \caption{Afzonderlijke features of settings zijn dan weer logischer met toggle switches.}
\end{figure}

\begin{figure}[H]
    \centering
    \includegraphics[width=0.5\textwidth]{toggle-ex-6.png}
    \caption{Een enkele ja/nee optie is logischer met een checkbox.}
\end{figure}

\begin{figure}[H]
    \centering
    \includegraphics[width=0.5\textwidth]{toggle-ex-7.png}
    \caption{Een enkele aan/af beslissing is logischer met een toggle switch.}
\end{figure}

\subsubsection{Toggle Switch of Toggle Button?}

\bold{Toggle buttons = soort van veredelde radio button of meerdere buttons}

\begin{itemize}
    \item Contextual state
    \item Heeft invloed op het huidige scherm
    \item Opposing options
\end{itemize}


\bold{Toggle switch = veredelde checkbox}
\begin{itemize}
    \item System state
    \item Heeft invloed op de volledige app
    \item Binary options
\end{itemize}

\begin{figure}[H]
    \centering
    \includegraphics[width=0.5\textwidth]{toggle-buttons.png}
    \caption{}
\end{figure}

\begin{figure}[H]
    \centering
    \includegraphics[width=0.5\textwidth]{toggle-buttons2.png}
    \caption{Switches are for binary options, not opposing options.}
\end{figure}

\subsection{Textarea}
\begin{itemize}
    \item <textarea>
    \item Geen input type, apart element
    \item Multi-line text input control
    \item De gebruiker kan optioneel de grootte aanpassen van het tekstvak
\end{itemize}

\subsection{Select}
\begin{itemize}
    \item <select>
    \item Geen input type, apart element
    \item dropdown list
    \item De <option> tags binnen het <select> element definieren de beschikbare options in de lijst
    \item \bold{Native select} behouden als je designt: geen eigen element proberen te maken $\Rightarrow$ gebruiksvriendelijker op smartphone.
\end{itemize}

\subsection{Range}
\begin{itemize}
    \item Slider control
    \item Voor nummers
    \item Default range van 0 tot 100
    \begin{itemize}
        \item restrictions zijn mogelijk met max, min en step attributes
    \end{itemize}
\end{itemize}

\subsection{Hors catégorie}
\begin{itemize}
    \item \bold{file} - file upload
    \item \bold{hidden} - voor developers
    \item \bold{color} - toont een color picker
\end{itemize}


\subsection{Attributes}
= Eigenschappen van de input

\begin{itemize}
    \item bv: type, value, id, name, \dots
    \item Alle attributes: \url{https://www.w3schools.com/html/html_form_attributes.asp}
\end{itemize}

\subsubsection{Minimum attributes}
\begin{itemize}
    \item \bold{type} - defineert het type input (duh!)
    \item \bold{name}
    \begin{itemize}
        \item Voor developers
        \item Zorgt er voor dat je weet wat je waar ingevuld hebt
        \item Ook om radio buttons aan elkaar te koppelen
        \item \bold{Dus altijd een name voorzien!}
    \end{itemize}
\end{itemize}

\subsubsection{Veelgebruikte attributes}

\begin{itemize}
    \item \bold{value}
    \begin{itemize}
        \item Kan je gebruiken om een default value in te geven
        \item Indien leeg wordt dit wat je hebt ingevuld
    \end{itemize}
    \item \bold{placeholder}
    \begin{itemize}
        \item hint
        \item voorbeeld van de wat de value moet zijn
        \item Verdwijnt automatisch als je begint te typen
        \item Placeholder nooit gebruiken als alternatief voor een label!
    \end{itemize}
\end{itemize}

\begin{figure}[H]
    \centering
    \includegraphics[width=0.5\textwidth]{veelgebruikte-attributes.png}
    \caption{}
\end{figure}

\subsubsection{Lege attributes}

\begin{itemize}
    \item Hebben geen value
    \item Zijn waar of onwaar
    \item CSS pseudo classes
    \item Veel gebruikt:
    \begin{itemize}
        \item \bold{Checked} - radio options \& checkboxes
        \item \bold{Required} - voor browservalidatie
        \item \bold{Disabled} - kan je niet aanpassen en wordt ook niet gesubmit
        \item \bold{Readonly} - kan je niet aanpassen maar wordt wel gesubmit
        \item \bold{Autofocus} - focust automatisch op het input type
    \end{itemize}
\end{itemize}

\begin{figure}[H]
    \centering
    \includegraphics[width=0.5\textwidth]{lege-attributes.png}
    \caption{required}
\end{figure}


\subsection{Labels}
\begin{itemize}
    \item Gekoppeld aan een input
    \item Usability improvement: toggles the input
    \item Elke input moet een label hebben!
    \item Niet vervangen door placeholder!
    \begin{itemize}
        \item In een lang ingevuld formulier weet je op den duur niet meer wat je waar moet invullen
        \item Alternatief: floating label pattern:
        \item \url{https://dribbble.com/shots/3429471-Floating-label-input-field}
        \item \url{https://codepen.io/soulrider911/pen/ugnyl}
    \end{itemize}
\end{itemize}

\subsubsection{Label koppelen aan input - Manier 1}

\begin{itemize}
    \item Met for en id
    \item Label text is apart aanspreekbaar met CSS.
\end{itemize}

\begin{lstlisting}[language=HTML]
<label for="input_id">label text</label>
<input type="text" name="whatever" id="input_id">
\end{lstlisting}



\subsubsection{Label koppelen aan input - Manier 2}
\begin{itemize}
    \item Geen for en id nodig (maar wel altijd aangeraden)
    \item Label text is niet aanspreekbaar met CSS.
\end{itemize}

\begin{lstlisting}[language=HTML]
    <label>
        label text
        <input type="text" name="whatever" id="input_id">
    </label>
    \end{lstlisting}


\subsection{Buttons}
\bold{Elk formulier moet een button hebben! (Binnen de form tag)}

\subsubsection{Als input type}

\begin{lstlisting}[language=HTML]
<input type="submit" value="verzenden">
<input type="button" value="verzenden">
\end{lstlisting}

\subsubsection{Als button element}

\begin{lstlisting}[language=HTML]
<button>verzenden</button>
\end{lstlisting}

\bold{Gebruik het button element!}

\begin{lstlisting}[language=HTML]
<button class="c-button">
    <span class="c-button__label">Verzenden</span>
    <svg class="c-button__symbol>...</svg>
</button>
\end{lstlisting}

\subsection{States}
\begin{enumerate}
    \item :hover
    \item :active
    \item :focus
\end{enumerate}

Deze volgorde in de CSS is zeer belangrijk!

\url{https://codepen.io/simoncoudeville-nmct/pen/oNxJbea}

\subsubsection{:hover}

CSS declarations worden geactiveerd\dots

\begin{itemize}
    \item Op pc: wanneer een gebruiker de muis over een element beweegt.
    \item Op mobiele toestellen: als een gebruiker een element "induwt" en "loslaat" en er geen specifieke focus of active styles zijn gedeclareerd of als :hover na :focus of :active komt.
\end{itemize}

\subsubsection{:active}

\begin{itemize}
    \item CSS declarations worden geactiveerd wanneer een gebruiker met de muis klikt.
    \item CSS declarations worden geactiveerd op mobiele toestellen als een gebruiker een element "induwt".
    \item Extra feedback feedback
\end{itemize}

\subsubsection{:focus}
\begin{itemize}
    \item Focus toont duidelijk welk element op de pagina keyboard events kan ontvangen
    \item Het element dat gefocust is heeft een duidelijke focus ring of outline of een andere visuele clue die de designer voorzien heeft.
    \item Welk element gefocust is kan je bedienen met het keyboard via tab of shift tab
    \item De volgorde is de tab order
    \item Interactieve HTML elementen zoals input, buttons, links zijn impliciet focusbaar. Ze worden automatisch aan de tab order toegevoegd.
    \item Paragrafen, divs, images enz\dots zijn niet focusbaar
\end{itemize}

\begin{figure}[H]
    \centering
    \includegraphics[width=0.4\textwidth]{focus.png}
    \includegraphics[width=0.2\textwidth]{focus2.png}
    \caption{}
\end{figure}

\begin{itemize}
    \item Voorbeeld van een form waar je niet op kan klikken, te bedienen met de tab-toets
    \item \url{http://udacity.github.io/ud891/lesson2-focus/01-basic-form/}
    \item Probeer eens een ticket te boeken voor\dots
    \begin{itemize}
        \item een round trip
        \item van Sydney naar Melbourne
        \item van 12 oktober tot 23 oktober 2018
        \item aan het venster
        \item en je wil geen promotionele aanbiedingen
    \end{itemize}
    \item Met tab, de pijltjes, spatie voor checkbox, \dots
\end{itemize}

\subsection{Validation}

\begin{itemize}
    \item Voorkom dat gebruikers fouten maken
    \item Als ze dan toch fouten maken en kunnen submitten:
    \begin{itemize}
        \item Verzorg duidelijke foutboodschappen
        \item Zet foutboodschappen inline bij hun input
    \end{itemize} 
\end{itemize}

\subsubsection{Client side validation}
\begin{itemize}
    \item Voor dat data wordt doorgestuurd naar de server
    \item Instant response
    \item HTML5 validation
    \item \begin{itemize}
        \item Required, valid \& invalid pseudo classes om instant te tonen of het juist is of niet.
        \item \url{https://codepen.io/chriscoyier/pen/JXgKjb}
    \end{itemize}
    \item + Javascript validation omdat je html kan aanpassen in developers tools
\end{itemize}

\subsubsection{Server side validation}
\begin{itemize}
    \item Voordat het opgeslaan wordt in de database
    \item Laatste
\end{itemize}

\subsubsection{Voorbeelden}
\begin{figure}[H]
    \centering
    \includegraphics[width=0.45\textwidth]{validation1.png}
    \includegraphics[width=0.45\textwidth]{validation2.png}
    \caption{}
\end{figure}

\begin{figure}[H]
    \centering
    \includegraphics[width=0.35\textwidth]{validation3.png}
    \caption{}
\end{figure}

\begin{figure}[H]
    \centering
    \includegraphics[width=0.45\textwidth]{validation4.png}
    \includegraphics[width=0.45\textwidth]{validation5.png}
    \caption{Waar validation gebruiken?}
\end{figure}


\subsection{Extra (geen vragen op examen)}
\subsubsection{States}
\url{https://zellwk.com/blog/style-hover-focus-active-states/}


\subsubsection{HTML5 form validation}
\url{https://developer.mozilla.org/en-US/docs/Learn/HTML/Forms/Form_validation}

\subsubsection{Best practices}

\begin{itemize}
    \item \url{https://www.smashingmagazine.com/2018/08/best-practices-for-mobile-form-design}
    \item \url{https://uxplanet.org/10-rules-for-efficient-form-design-e13dc1fb0e03}
    \item \url{https://uxmovement.com/mobile/stop-misusing-toggle-switches}
    \item \url{https://www.bram.us/2019/01/18/building-better-forms-by-not-taking-away-affordances}
\end{itemize}

\end{document}


\documentclass{article}

\usepackage[dutch]{babel}
\usepackage[margin=3cm]{geometry}
\usepackage{graphicx}
\usepackage{float}
\usepackage{caption}
\usepackage{hyperref}
\usepackage{amsmath}
\usepackage{wrapfig}
\usepackage[parfill]{parskip}

% fonts
\usepackage[T1]{fontenc}
\usepackage{helvet}
\renewcommand{\familydefault}{\sfdefault}

\graphicspath{{img/}}
 
\newcommand{\bold}[1]{\textbf{#1}}

%Define the listing package
\usepackage{listings} %code highlighter
\usepackage{upquote}
\usepackage{color} %use color
\definecolor{mygreen}{rgb}{0,0.6,0}
\definecolor{mygray}{rgb}{0.5,0.5,0.5}
\definecolor{mymauve}{rgb}{0.58,0,0.82}
 
%Customize a bit the look
\lstset{ %
backgroundcolor=\color{white}, % choose the background color; you must add \usepackage{color} or \usepackage{xcolor}
basicstyle=\footnotesize, % the size of the fonts that are used for the code
breakatwhitespace=false, % sets if automatic breaks should only happen at whitespace
breaklines=true, % sets automatic line breaking
captionpos=b, % sets the caption-position to bottom
commentstyle=\color{mygreen}, % comment style
deletekeywords={...}, % if you want to delete keywords from the given language
escapeinside={\%*}{*)}, % if you want to add LaTeX within your code
extendedchars=true, % lets you use non-ASCII characters; for 8-bits encodings only, does not work with UTF-8
frame=single, % adds a frame around the code
keepspaces=true, % keeps spaces in text, useful for keeping indentation of code (possibly needs columns=flexible)
keywordstyle=\color{blue}, % keyword style
% language=Octave, % the language of the code
morekeywords={*,...}, % if you want to add more keywords to the set
numbers=left, % where to put the line-numbers; possible values are (none, left, right)
numbersep=5pt, % how far the line-numbers are from the code
numberstyle=\tiny\color{mygray}, % the style that is used for the line-numbers
rulecolor=\color{black}, % if not set, the frame-color may be changed on line-breaks within not-black text (e.g. comments (green here))
showspaces=false, % show spaces everywhere adding particular underscores; it overrides 'showstringspaces'
showstringspaces=false, % underline spaces within strings only
showtabs=false, % show tabs within strings adding particular underscores
stepnumber=1, % the step between two line-numbers. If it's 1, each line will be numbered
stringstyle=\color{mymauve}, % string literal style
tabsize=2, % sets default tabsize to 2 spaces
title=\lstname % show the filename of files included with \lstinputlisting; also try caption instead of title
}
%END of listing package%

\lstdefinelanguage{CSS}{
      keywords={accelerator,azimuth,background,background-attachment,
            background-color,background-image,background-position,
            background-position-x,background-position-y,background-repeat,
            behavior,border,border-bottom,border-bottom-color,
            border-bottom-style,border-bottom-width,border-collapse,
            border-color,border-left,border-left-color,border-left-style,
            border-left-width,border-right,border-right-color,
            border-right-style,border-right-width,border-spacing,
            border-style,border-top,border-top-color,border-top-style,
            border-top-width,border-width,bottom,caption-side,clear,
            clip,color,content,counter-increment,counter-reset,cue,
            cue-after,cue-before,cursor,direction,display,elevation,
            empty-cells,filter,float,font,font-family,font-size,
            font-size-adjust,font-stretch,font-style,font-variant,
            font-weight,height,ime-mode,include-source,
            layer-background-color,layer-background-image,layout-flow,
            layout-grid,layout-grid-char,layout-grid-char-spacing,
            layout-grid-line,layout-grid-mode,layout-grid-type,left,
            letter-spacing,line-break,line-height,list-style,
            list-style-image,list-style-position,list-style-type,margin,
            margin-bottom,margin-left,margin-right,margin-top,
            marker-offset,marks,max-height,max-width,min-height,
            min-width,-moz-binding,-moz-border-radius,
            -moz-border-radius-topleft,-moz-border-radius-topright,
            -moz-border-radius-bottomright,-moz-border-radius-bottomleft,
            -moz-border-top-colors,-moz-border-right-colors,
            -moz-border-bottom-colors,-moz-border-left-colors,-moz-opacity,
            -moz-outline,-moz-outline-color,-moz-outline-style,
            -moz-outline-width,-moz-user-focus,-moz-user-input,
            -moz-user-modify,-moz-user-select,orphans,outline,
            outline-color,outline-style,outline-width,overflow,
            overflow-X,overflow-Y,padding,padding-bottom,padding-left,
            padding-right,padding-top,page,page-break-after,
            page-break-before,page-break-inside,pause,pause-after,
            pause-before,pitch,pitch-range,play-during,position,quotes,
            -replace,richness,right,ruby-align,ruby-overhang,
            ruby-position,-set-link-source,size,speak,speak-header,
            speak-numeral,speak-punctuation,speech-rate,stress,
            scrollbar-arrow-color,scrollbar-base-color,
            scrollbar-dark-shadow-color,scrollbar-face-color,
            scrollbar-highlight-color,scrollbar-shadow-color,
            scrollbar-3d-light-color,scrollbar-track-color,table-layout,
            text-align,text-align-last,text-decoration,text-indent,
            text-justify,text-overflow,text-shadow,text-transform,
            text-autospace,text-kashida-space,text-underline-position,top,
            unicode-bidi,-use-link-source,vertical-align,visibility,
            voice-family,volume,white-space,widows,width,word-break,
            word-spacing,word-wrap,writing-mode,z-index,zoom},  
      sensitive=true,
      morecomment=[l]{//},
      morecomment=[s]{/*}{*/},
      morestring=[b]',
      morestring=[b]",
      alsoletter={:},
      alsodigit={-}
    }

\begin{document}

\begin{titlepage}
    \author{Tuur Vanhoutte}
    \title{Interaction design}
\end{titlepage}

\pagenumbering{gobble}
\maketitle
\newpage
\tableofcontents
\newpage

\pagenumbering{arabic}

\section{CSS Variables}

\subsection{Wat?}

\begin{itemize}
    \item CSS custom properties for cascading variables.
    \item CSS Variables = Custom Properties.
    \item Relatief nieuwe manier om veelgebruikte values om te zetten naar Variables.
    \item To keep consistency, set global variables for everything except layout values.
\end{itemize}

\subsection{Opbouw \& Syntax}

\subsubsection{Custom property: defining the variable}
Custom property: value

\begin{lstlisting}[language=CSS]
--my-cool-color: HotPink;
\end{lstlisting}[language=CSS]

\subsubsection{Cascading variable: applying the variable}

Applying your custom property using the var() function

\begin{lstlisting}[language=CSS]
var(--my-cool-color)
\end{lstlisting}[language=CSS]

\subsection{Syntax}

\begin{lstlisting}[language=CSS]
:root {
    --my-cool-color: HotPink;
}

p {
    color: var(--my-cool-color);
}

.foo {
    background-color: var(--my-cool-color);
}
\end{lstlisting}[language=CSS]

\subsubsection{CSS Variables Are Case Sensitive}

\begin{lstlisting}[language=CSS]
:root {
    --foo: HotPink;
    --FOO: #BADA55;
}
p {
    color: var(--foo);
}
.foo {
    background-color: var(--FOO);
}
\end{lstlisting}[language=CSS]


\subsubsection{This is wrong}

\begin{lstlisting}[language=CSS]
.test {
    --side: margin-top;
    var(--side): 20px
}
\end{lstlisting}[language=CSS]

\subsubsection{Use the calc() function to do math}

\begin{lstlisting}[language=CSS]
:root {
    --whitespace: 20px;
    --whitespace-lg: var(--whitespace) * 2; /* won't work */
    --whitespace-lg: calc(var(--whitespace) * 2); /* correct */
}
\end{lstlisting}[language=CSS]

\subsubsection{Kan ook shorthand values bevatten}

\begin{lstlisting}[language=CSS]
:root {
    --transition: all .1s ease-out;
}
a {
    transition: var(--transition);
}   
\end{lstlisting}[language=CSS]

\subsubsection{Kan ook bestaan uit andere variables.}

\begin{lstlisting}[language=CSS]
:root {
    --transition-property: all;
    --transition-duration: .1s;
    --transition-timing-function: ease-out;
    --transition: var(--transtion-property) 
        var(--transition-duration) 
        var(--transition-timing-function);
}

a {
    transition: var(--transition);
}
\end{lstlisting}[language=CSS]

\subsubsection{Default values}

\begin{lstlisting}[language=CSS]
.c-button {
    border: 1px solid var(--button-color, HotPink);
    background-color: transparent;
}
.c-button:hover {
    background-color: var(--button-color, HotPink);
}
.c-button--beta {
    --button-color: #BADA55;
}
\end{lstlisting}[language=CSS]

\subsubsection{Default values with other variables}

\begin{lstlisting}[language=CSS]
:root {
    --color-pink: HotPink;
    --color-badass: #BADA55;
}
.c-button {
    border: 1px solid var(--button-color, var(--color-pink));
}
.c-button:hover {
    background-color: var(--button-color, var(--color-pink));
}
.c-button--beta {
    --button-color: var(--color-badass);
\end{lstlisting}[language=CSS]

\subsection{Cascade}
CSS Variables Are Subject to the Cascade and Inheritance rules

\url{https://codepen.io/simoncoudeville-nmct/pen/BaBMGPZ}

\begin{lstlisting}[language=CSS]
:root {
    --my-cool-color: HotPink;
}
/* Alle elementen binnen de root waar je --
my-cool-color variable toepast zullen de
value HotPink overerven. */

p {
color: var(--my-cool-color);
}

.foo {
/* Elke paragraaf in het element met
de class ".foo" krijgt de kleur #BADA55.*/

--my-cool-color: #BADA55;
}
\end{lstlisting}[language=CSS]

\url{https://codepen.io/simoncoudeville-nmct/pen/aboMBop}

\subsubsection{CSS variables can be made conditional with @media and other conditional rules}

\begin{lstlisting}[language=CSS]
:root {
    --whitespace: 1em;
}
@media screen and (min-width: 768px) {
    :root {
        --whitespace: 2em;
    }
}
.c-card {
    padding: var(--whitespace);
}

\end{lstlisting}[language=CSS]


\url{https://codepen.io/simoncoudeville-nmct/pen/JjXaQwB}

\subsubsection{Ideal for dark themes}

\begin{lstlisting}[language=CSS]
:root {
    --color: white;
    --background-color: black
}
@media (prefers-color-scheme: dark) {
    :root {
    --color: black;
    --background-color: white
    }
}
.html {
    background-color: var(--background-color);
    color: var(--color);
}
\end{lstlisting}[language=CSS]

\url{https://codepen.io/simoncoudeville-nmct/pen/eYOxbPp}

\subsection{Hoisting}
= Accessing a Variable First and Declaring Later

\begin{lstlisting}[language=CSS]
body{
    background: var(--bg-fill);
}
:root{
    --bg-fill: green;
}

\end{lstlisting}[language=CSS]

\subsection{Scoped variables}

Twee soorten:
\begin{itemize}
    \item Global Scoped Variables
    \item Local Scoped Variables
\end{itemize}

\subsubsection{Global scoped variables}

\begin{lstlisting}[language=CSS]
:root {
    --global-color: black;
}
\end{lstlisting}[language=CSS]
\begin{itemize}
    \item :root is a CSS pseudo-class selector used to select the element that represents the root of the document.
    \item :root is hetzelfde als html maar is specifieker
    \item :root = html
    \item :root > html
\end{itemize}

\url{https://codepen.io/simoncoudeville-nmct/pen/OJLBKXq}

Global variables kan je overal hergebruiken en overschrijven. Ideaal dus voor values die veel hergebruikt worden:

\begin{itemize}
    \item colors
    \item whitespace
    \item border-radius
    \item transitions
    \item \dots
\end{itemize}

\subsubsection{Local scoped variables}
\begin{itemize}
    \item Local scoped variables worden gedeclareerd binnen een specifieke selector.
    \item Local scoped variables hebben access tot global scoped variables.
    \item Ideaal voor components.
\end{itemize}

\begin{lstlisting}[language=CSS]
.alert {
    --alert-color: #222;
    color: var(--alert-color);
    border-color: var(--alert-color);
}
:root {
    --global-fontSize: 16px;
}
.c-button {
    --button-fontSize: var(--global-fontSize);
    font-size: var(--button-fontSize);
}
\end{lstlisting}[language=CSS]


\subsection{Naming system}

\subsubsection{The two-level theming system}

Systeem om global variables te gebruiken in local variables.

\subsubsection{Global level}

De hoofdreden om global variables te hebben is consistentie.

They are prefixed with the word global and follow this formula:

-{}-global-{}-concept-{}-modifier-{}-state-{}-propertyCamelCase

\begin{itemize}
    \item a concept is something like a spacer, main-title or text
    \item a state is something like hover, or expanded
    \item a modifier is something like sm, or lg
    \item and a propertyCamelCase is something like backgroundColor or fontSize
\end{itemize}

\subsubsection{Local level}

They follow this formula: 

-{}-block\_\_element-{}-modifier-{}-state-{}-propertyCamelCase

\begin{itemize}
    \item The block\_\_element-{}-modifier selector name is something like alert\_\_actions or alert--primary
    \item a state is something like hover or active
    \item The value of component scoped variables is always defined by a global variable.
\end{itemize}


\begin{lstlisting}[language=CSS]
.c-alert {
    /* Component scoped variables are always defined by global variables */
    --c-alert--Padding: var(--global--spacer--md);
    --c-alert--primary--BackgroundColor: var(--global--primary-color);
    --c-alert__title--FontSize: var(--global--secondary-title--fontSize);
    /* --block--propertyCamelCase */
    padding: var(--c-alert--padding);
}

/* --block--state--propertyCamelCase */
.c-alert--primary {
    background-color: var(--c-alert--primary--backgroundColor);
}

/* --block__element--propertyCamelCase */
.c-alert__title {
    font-size: var(--c-alert__title--fontSize);
}
\end{lstlisting}[language=CSS]

\subsection{Herhalingsvragen}

\begin{itemize}
    \item Hoe worden CSS variables nog genoemd?
    \item Hoe worden CSS variables opgebouwd?
    \item Wat is CSS Hoisting?
    \item Wat is het verschil tussen global en local variables?
\end{itemize}

\url{https://codepen.io/simoncoudeville-nmct/pen/vYBbbXz?editors=1100}

\end{document}
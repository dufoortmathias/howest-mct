\documentclass{article}

\usepackage[english]{babel}
\usepackage[margin=3cm]{geometry}
\usepackage{graphicx}
\usepackage{float}
\usepackage{caption}
\usepackage{hyperref}
\usepackage{amsmath}
\usepackage{wrapfig}
\usepackage[parfill]{parskip}

% fonts
\usepackage[T1]{fontenc}
\usepackage{helvet}
\renewcommand{\familydefault}{\sfdefault}

\graphicspath{{img/}}

% theorem environment
\usepackage{amssymb}

\newtheorem{theorem}{Definitie}[section]

\usepackage{enumitem}

\newenvironment{thmenum}
 {\begin{enumerate}[label=\upshape\bfseries(\roman*)]}
 {\end{enumerate}}


% code
\usepackage{minted}
\usepackage{upquote}
\usepackage{color}

\begin{document}

\begin{titlepage}
    \author{Tuur Vanhoutte}
    \title{Linux OS}
\end{titlepage}

\pagenumbering{gobble}
\maketitle
\newpage
\tableofcontents
\newpage

\pagenumbering{arabic}

\section{Introductie}

\subsection{Verschil Server \& Workstation}

\subsubsection{Server}

\begin{itemize}
    \item Deliver services to (multiple) users
    \item Focussed: only this and nothing else
    \item Secure
    \item No GUI, everything happens through the commandline
    \item $\Rightarrow$ as small a footprint as possible
\end{itemize}

\subsubsection{Workstation}

\begin{itemize}
    \item Use services
    \item Create documents
    \item Look for information
    \item Consume multimedia
    \item GUI
    \item $\Rightarrow$ Large footprint
\end{itemize}

\subsection{Extra information/resources}

\begin{itemize}
    \item The Linux Documentation Project: \url{http://tldp.org}
    \item Pluralsight LPIC-1: Linux Professional Institute Certification: \url{https://www.pluralsight.com/paths/lpic-1}
    \item The Arch Linux Wiki is one of the most extensive sources of info about Linux: \url{https://wiki.archlinux.org}
    \begin{itemize}
        \item In this module we will use Debian, not Arch, but many things are very similar
    \end{itemize}
    \item Google
\end{itemize}

\subsection{What is Linux?}

\subsubsection{What is an operating system (OS)?}

\begin{theorem}[Operating System]
An operating system, or OS, is software that communicates with the hardware
and alows other programs to run. 

It is comprised of system software = the fundamental files your computer needs to function.
\end{theorem}


\textcolor{red}{Linux is NOT an operating system: Linux = the \underline{\textbf{kernel}}}

\subsubsection{What is a Kernel?}

\begin{theorem}[Kernel]
The kernel is software that is the core of a computer's operating system, with complete control over the system.

It is the first program loaded on start-up. 

It handles\dots: 

\begin{itemize}
    \item \dots the rest of the startup
    \item \dots input/output requests from software, translating them into instructions for the CPU
    \item \dots memory
    \item \dots peripherals
\end{itemize}
\end{theorem}


\subsection{GNU Operating System}

\begin{theorem}[GNU]
GNU = GNU's Not Unix (recursive algorithm)

Founded by Richard Stallman (ex-MIT, founder of the Free Software Foundation), 1984

Goal: completely free Operating System
\end{theorem}

\subsection{Linux, the kernel}

By Linus Torvalds (Finland), 1991

\begin{itemize}
    \item Own personal development, not initially intended to distribute
    \item Interest from other developers, mainly to use with GNU OS
    \item Meanwhile contributions of over 12000+ developers
    \item 492 of top-500 supercomputers in the world run Linux
    \item Basis for Android, Chrome OS
\end{itemize}

Linux = the kernel

GNU = OS-tools around the kernel

$\Rightarrow$ \textbf{GNU/Linux}

\subsubsection{Distributions}

\begin{theorem}[Distribution]
A Linux distribution (or distro for short) is GNU/Linux + extra tools and applications to create a full-fledged OS.

That distribution can be easily copied and installed to other computers.
\end{theorem}

\begin{itemize}
    \item RedHat (CentOS)
    \item Debian (Ubuntu)
    \item Arch Linux
    \item Void Linux
    \item Gentoo
    \item Pop! OS
\end{itemize}

\begin{figure}[H]
    \centering
    \includegraphics[width=0.75\textwidth]{redhat-vs-debian-tree.png}
    \caption{\url{https://upload.wikimedia.org/wikipedia/commons/1/1b/Linux_Distribution_Timeline.svg}}
\end{figure}

\url{https://en.wikipedia.org/wiki/List_of_Linux_distributions}

\subsection{Open Source}

\begin{theorem}[Open Source]
Open source software is software of which the code is licensed to be open to everyone. 

Anyone can use, change, distribute the software. This allows code to be developed in a public manner.
\end{theorem}

\textbf{\textcolor{red}{OPEN SOURCE DOES NOT MEAN FREE}}

\subsubsection{Commercial distributions}

= Open source, non-free distributions

\begin{itemize}
    \item SUSE Linux Enterprise Server (SLES)
    \item SUSE Linux Enterprise Desktop (SLED)
    \item Red Hat Enterprise Linux (RHEL)
    \item Oracle Enterprise Linux
\end{itemize}

Commercial distributions have official support channels.

$\Rightarrow$ You're not paying for the operating system, you're paying for the support.

\subsubsection{In this course: Debian}


\begin{itemize}
    \item Current version: 10.7
    \item Forms the basis of many others: Ubuntu, Raspbian, Knoppix, Linux Mint
    \item Available on many platforms: Intel x86, AMD64, Intel64, ARM, MIPS, Power Systems, \dots
\end{itemize}

\section{Debian Installation}

\textbf{See Labs for detailed Installation tutorial}

\subsection{Networking in Linux (with VMWare)}

\begin{itemize}
    \item VMWare presents ethernet adapter
    \item During creation of virtual machine: MAC-address is created
    \item During installation: network configuration through DHCP
    \begin{itemize}
        \item IPv4-address
        \item Default gateway
        \item DNS-server
        \item \underline{Optional:} proxy-server
    \end{itemize}
\end{itemize}

\subsection{Users in Linux}

\begin{itemize}
    \item Linux is multi-user from the ground up
    \begin{itemize}
        \item Multiple users can be active at the same time
    \end{itemize}
    \item `Administrator'-user is called root
    \item Each user has a user-ID (uid)
    \begin{itemize}
        \item root has uid=0
        \item uid=0 has all rights
    \end{itemize}
    \item Each user has a home-directory
\end{itemize}

\subsection{Disks, partition, filesystems}

\begin{itemize}
    \item Our VM has 1 disk
    \begin{itemize}
        \item Presented on the SCSI-bus
        \item First disk on SCSI-bus: \textbf{sda}
        \item Then sdb, sdc, \dots
    \end{itemize}
    \item Disk = concatenation of blocks
    \item Divide blocks in collections (=partitions)
    \begin{itemize}
        \item 1st partition: sda1
        \item 2nd partition: sda2
        \item \dots
    \end{itemize}
    \item 2 types of partitions
    \begin{itemize}
        \item Primary
        \item Extended
    \end{itemize}
\end{itemize}

\subsubsection{Partitions}

Primary partition

\begin{itemize}
    \item A filesystem can be created inside this
    \item Up to 4 primary partitions
\end{itemize}

Extended Partition

\begin{itemize}
    \item `Logical' partitions can be created inside this
\end{itemize}

Our setup:

\begin{itemize}
    \item sda1: primary partition
    \item sda2: extended partition
    \item sda5: `logical' partition inside extended partition sda2
\end{itemize}

\begin{figure}[H]
    \centering
    \includegraphics[width=0.3\textwidth]{partitions.png}
    \caption{Our setup}
\end{figure}

\subsection{MBR <> GPT}

\subsubsection{MBR}

We use the MBR Partitioning scheme

\begin{theorem}[MBR]
MBR, or Master Boot Record, is a special type of boot sector at the start of a disk.

It contains: 

\begin{itemize}
    \item a set of instructions necessary to boot operating systems.
    \item info about how partitions are placed on disk
\end{itemize}


\end{theorem}

Limitations:

\begin{itemize}
    \item Maximum disks of 2TB
    \item 32-bit for number of logical sectors
    \item Common sector size: 512 bytes
    \item $2^{32} \cdot 512\ \text{bytes} = 4294967296 \cdot 512\ \text{bytes} \approx 2\text{TB}$
\end{itemize}

BIOS can boot from a disk with MBR partitioning

\subsubsection{GPT}

\begin{theorem}[GPT]
GPT, or GUID Partition Table, is a standard for the layout of partition tables on a disk.
It's an alternative to MBR.

It uses unique identifiers (GUIDs) 
\end{theorem}

\begin{itemize}
    \item BIOS cannot boot from a disk with GPT-partitioning: UEFI required when using GPT
    \item GPT allows disks larger than 2TB
\end{itemize}

\begin{theorem}[UEFI]
UEFI, or Unified Extensible Firmware Interface, is a newer firmware interface by Intel (90's) that replaces the BIOS interface by IBM (70's).
\end{theorem}

\textbf{How does it work?}

\begin{itemize}
    \item Disk = collection of blocks
    \item Group of blocks together = sector
    \item Common sector size: 512 bytes
    \item Sectors indicated with Logical Block Addresses (LBA)
    \item MBR in LBA 0
    \item GPT headers in LBA 1
    \item Partition tabel right after that
\end{itemize}




\subsubsection{Bootstrap procedure}

\begin{enumerate}
    \item Motherboard gets electricity
    \item Mini-loader hardcoded in memory
    \begin{itemize}
        \item BIOS gets loaded
    \end{itemize}
    \item Boot media are consulted
    \item First boot medium, first sectors are being read $\Rightarrow$
    \item MBR contains a bit-more-advanced loader: GRUB
    \begin{itemize}
        \item \underline{GR}and \underline{U}nified \underline{B}ootloader
    \end{itemize}
    \item This loader loads a more advanced loader (GRUB second stage bootloader)
    \item The OS is loaded
\end{enumerate}

\subsubsection{Linux boot process}

6 high level steps

\begin{itemize}
    \item BIOS (Basic Input/Output System) - loads MBR
    \item MBR (Master Boot Record) - loads GRUB
    \item GRUB (Grand Unified Bootloader) - loads kernel
    \item Kernel - executes /sbin/init
    \item Init - executes runlevel programs
    \item Runlevel - programs from /etc/rc.d/rcXX.d are started
\end{itemize}

\subsubsection{BIOS <> UEFI}

\begin{itemize}
    \item Recent systems use UEFI, not BIOS
    \item UEFI is required to boot from GPT-disk
    \item Linux has no trouble working with UEFI
\end{itemize}

\textbf{So why will we use MBR?}

\begin{itemize}
    \item Virtualisation is the norm
    \item Virtual machines typically have small disks
    \item Small disks are MBR partitioned
\end{itemize}



\subsection{Filesystems}

\subsubsection{Windows}

\begin{itemize}
    \item FAT (1977)
    \item FAT32 (1996)
    \item NTFS (1993)
    \item ReFS (2012)
\end{itemize}

\subsubsection{Linux}

\begin{itemize}
    \item Ext (1992)
    \item Ext2 (1993)
    \item Ext3 (2001)
    \item Ext4 (2008)
    \item ZFS (2005)
    \item BtrFS (2007)
\end{itemize}

\subsubsection{Swap}

= Paging

\begin{itemize}
    \item Free up physical memory (RAM) by moving pages to slower storage (storage disks instead of RAM)
    \item Page out $=$ memory page moves to swap
    \item "Swapiness"
    \begin{itemize}
        \item = parameter between 0 and 100
        \item = how quickly linux will swap
        \begin{itemize}
            \item 0 = very conservative
            \item 100 = very agressive
        \end{itemize}
    \end{itemize}
    \item Windows uses a swap file (pagefile.sys)
    \item Linux uses a swap partition
\end{itemize}

\subsection{File structure}



\begin{figure}[H]
    \centering
    \includegraphics[width=0.4\textwidth]{linux-filestructure.png}
    \caption{Linux uses a tree structure}
\end{figure}

\begin{figure}[H]
    \centering
    \includegraphics[width=0.5\textwidth]{windows-filestructure.png}
    \caption{Windows uses a similar structure, but every volume uses a letter.}
\end{figure}

\begin{figure}[H]
    \centering
    \includegraphics[width=0.5\textwidth]{linux-filestructure2.png}
    \caption{With linux, volumes are `mounted' to folders somewhere under root /}
\end{figure}

\subsection{Configuration}

\subsubsection{Packages}

\begin{itemize}
    \item Tools and applications are build up by files
    \item All files belonging to 1 application are bundled in a package
    \item Packages in debian have the .deb extension
\end{itemize}

Repositories

\begin{itemize}
    \item Packages are collected in repositories
    \item Are made available through the internet
    \item Packages have dependencies
\end{itemize}

\subsubsection{Package management}

Debian: dpkg \& apt (Advanced Package Tool)

\begin{itemize}
    \item dpkg: Install, remove, give info about .deb packages
    \begin{itemize}
        \item dpkg -l = lists packages 
    \end{itemize}
    \item apt: Get packages from a repository and install, remove, give info, ...
    \begin{itemize}
        \item apt update
        \begin{itemize}
            \item Contact the repositories
            \item Get most recent list of packages and versions
        \end{itemize}
        \item apt upgrade
        \begin{itemize}
            \item Of the packages which are more recent in the repositories compared to what is installed: install newest version
        \end{itemize}
        \item apt install <xyz>
        \begin{itemize}
            \item Download package <xyz> from the repository
            \item Check the dependencies and download depending packages
            \item Install package <xyz> and all corresponding dependencies
        \end{itemize}
    \end{itemize}
\end{itemize}

Which repositories? See /etc/apt/sources.list for the list of repositories. You can add/remove/change repositories in this file.

\subsubsection{Useful packages}

\begin{itemize}
    \item open-vm-tools
    \item vim
    \item sudo
    \item tcpdump
\end{itemize}

Install multiple pacakges in one command: apt install vim sudo tcpdump ntp

\subsection{Shutdown of VM}

\begin{itemize}
    \item Power button (=ACPI shutdown)
    \item Shut down operating system only
    \begin{itemize}
        \item = halt
    \end{itemize}
    \item Shut down operating system and VM, multiple ways:
    \begin{itemize}
        \item shutdown -P now
        \item init 0
        \item poweroff
    \end{itemize}
    \item Reboot
    \begin{itemize}
        \item reboot
        \item init 6
        \item shutdown -r now
    \end{itemize}
\end{itemize}

\subsection{Basic network}

\begin{itemize}
    \item No GUI $\Rightarrow$
    \item Layer 1: Physical (VMWare virtual network)
    \item Layer 2: Datalink (Ethernet \& MAC address)
    \item Layer 3: Network (IPv4)
    \item Layer 4: Transport (Transport Control Protocol (TCP), User Datagram Protocol (UDP))
    \item Layer 5: Application (SSH, HTTP, \dots)
\end{itemize}

\subsubsection{Basic networking commands}

\begin{itemize}
    \item arp
    \item ping
    \item route
    \item bmon
\end{itemize}

\subsection{Services}

\begin{itemize}
    \item Processes that `listen' on the network
    \begin{itemize}
        \item TCP or UDP port
    \end{itemize}
    \item Overview of currently running / listening services: ss command
    \begin{itemize}
        \item ss -tulpn
        \item t: show TCP
        \item u: show UDP
        \item l: show listening
        \item p: show process ID
        \item n: no name-resolving
    \end{itemize} 
\end{itemize}


\end{document}
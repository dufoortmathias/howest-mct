\documentclass{article}

\usepackage[dutch]{babel}
\usepackage[margin=3cm]{geometry}
\usepackage{graphicx}
\usepackage{float}
\usepackage{caption}
\usepackage{hyperref}
\usepackage{amsmath}
\usepackage{wrapfig}
\usepackage[parfill]{parskip}

% fonts
\usepackage[T1]{fontenc}
\usepackage{helvet}
\renewcommand{\familydefault}{\sfdefault}

\graphicspath{{img/}}

% theorem environment
\usepackage{amssymb}

\newtheorem{theorem}{Definitie}[section]

\usepackage{enumitem}

\newenvironment{thmenum}
 {\begin{enumerate}[label=\upshape\bfseries(\roman*)]}
 {\end{enumerate}}


% code
\usepackage{minted}
\usepackage{upquote}
\usepackage{color}

\begin{document}

\begin{titlepage}
    \author{Tuur Vanhoutte}
    \title{Advanced Programming \& Maths}
\end{titlepage}

\pagenumbering{gobble}
\maketitle
\newpage
\tableofcontents
\newpage

\pagenumbering{arabic}

\section{Basisfuncties in de wiskunde}

\subsection{Functies}

\begin{theorem}[Re"ele functie]
Een re"ele functie is een relatie in $\mathbb{R}$ waarbij elke waarde $x$ hoogstens één beeldwaarde $f(x)$ heeft
\end{theorem}

\begin{figure}[H]
    \centering
    \includegraphics[width=0.5\textwidth]{reele-functie.png}
    \caption{Voorbeelden re"ele functies}
\end{figure}

\begin{theorem}
Voor elke functie geldt: er bestaat een \dots
    \begin{thmenum}
        \item \dots domein van de functie (domain)
        \item \dots beeld van de functie (range)
        \item \dots functievoorschrift van de functie
    \end{thmenum}
\end{theorem}

\begin{figure}[H]
    \centering
    \includegraphics[width=0.5\textwidth]{functie-domain-range.png}
    \caption{Domein, bereik, functievoorschrift}
\end{figure}

$f: \mathbf{domein} \rightarrow \mathbf{bereik}: x \rightarrow y = f(x)$

$f: \mathbb{R} \rightarrow \mathbb{R} : x \rightarrow y = x^3 - 4x$

\begin{theorem}
    Elke functie kan nulpunten hebben.
\end{theorem}

\begin{figure}[H]
    \centering
    \includegraphics[width=0.5\textwidth]{functie-nulpunten.png}
    \caption{$y=-x^3 + 4x$}
\end{figure}

Verloop van een functie wordt via een tekenschema verduidelijkt:

\begin{figure}[H]
    \centering
    \includegraphics[width=0.5\textwidth]{functie-tekenschema.png}
    \caption{Tekenschema}
\end{figure}

\subsection{Veelterm en veeltermfuncties}

\begin{theorem}[Veelterm]
\begin{equation}
    \begin{aligned}
        A(x) = a_nx^n + a_{n-1}x^{n-1} + a_{n-2}x^{n-2} + ... + a_{2}x^{2} + a_1x + a_0\\
        (a_n,a_{n-1},...,a_2,a_1,a_0 \in \mathbb{R})
    \end{aligned}
\end{equation}

\end{theorem}

\begin{theorem}[Veeltermfunctie]
\begin{equation}
    \begin{aligned}
        f(x) = a_nx^n + a_{n-1}x^{n-1} + a_{n-2}x^{n-2} + ... + a_{2}x^{2} + a_1x + a_0\\
        Graad\ van\ veelterm = n\ als\ a_n \neq 0
    \end{aligned}
\end{equation}
\end{theorem}

\subsection{Bijzondere veeltermfuncties}

\begin{itemize}
    \item Constante functie: $f(x) = 4$
    \item Lineaire functie: $f(x) = 4$
    \item Tweedegraadsfunctie: $f(x) = 3x^2 + 2x + 1$
    \item Derdegraadsfunctie: $f(x) = 5x^3 - 3x^2 + 2x - 1$
    \item Exponenti"ele functie: $f(x) = 2^x$
    \item Logaritmische functie: $(fx) = log_2(x)$
\end{itemize}

\subsubsection{Constante functie}

\begin{figure}[H]
    \centering
    \includegraphics[width=0.3\textwidth]{functie-constant.png}
    \caption{$y=4$}
\end{figure}

\subsubsection{Lineaire functie}

\begin{theorem}[Lineaire functie]
    \begin{equation}
        f(x) = ax + b
    \end{equation}


Voorbeeld: $f(x) = 3x + 6$
\end{theorem}

\begin{itemize}
    \item Betekenis van a: de richtingsco"effici"ent (rico)
    \item Betekenis van b: het snijpunt met de y-as
    \item Nulpunt: $f(x) = 0 \\ \Leftrightarrow 3x + 6 = 0 \\ \Leftrightarrow 3x = -6  \\ \Leftrightarrow x = -2$
\end{itemize}

\begin{figure}[H]
    \centering
    \includegraphics[width=0.3\textwidth]{functie-lineair2.png}
    \caption{Meerdere evenwijdige lineaire functies}
\end{figure}

Evenwijdige rechten als: als $a_1 = a_2$

Loodrechte rechten als: als $a_1 \cdot a_2 = -1$

\subsubsection{Tweedegraadsfunctie}

\begin{theorem}
\begin{equation}
    \begin{aligned}
        f(x) = ax^2 + bx + c,\\
        (a \neq 0)
    \end{aligned}
\end{equation}


\end{theorem}

\begin{figure}[H]
    \centering
    \includegraphics[width=0.2\textwidth]{functie-2degraad.png}
    \caption{$f(x) = x^2 - 2x - 3$}
\end{figure}

\begin{itemize}
    \item Betekenis van a: positief $\Rightarrow$ dalparabool, negatief $\Rightarrow$ bergparabool
    \item Nulpunten: via de discriminant berekenen:
\end{itemize}

\begin{theorem}[Discriminant]
Bij een tweedegraadsvergelijking is de discriminant:

\begin{equation}
    D = b^2 - 4ac
\end{equation}
\end{theorem}

\begin{itemize}
    \item Geval 1: $D > 0 \Rightarrow$ de functie heeft 2 nulpunten
    \item Geval 2: $D = 0 \Rightarrow$ de functie heeft 1 nulpunt
    \item Geval 3: $D < 0 \Rightarrow$ de functie heeft géén nulpunten
\end{itemize}

\begin{figure}[H]
    \centering
    \includegraphics[width=0.4\textwidth]{functie-2degraad3.png}
    \includegraphics[width=0.3\textwidth]{functie-2degraad2.png}
    \caption{De discriminant toont de nulpunten}
\end{figure}

\textbf{Nulpunten berekenen:} 

\begin{equation}
    x_{1,2} = \frac{-b \pm \sqrt{D}}{2a}
\end{equation}

\begin{figure}[H]
    \centering
    \includegraphics[width=0.5\textwidth]{functie-2degraad4.png}
    \caption{Symmetrieas: $x = \frac{-b}{2a}$}
\end{figure}

Voorbeeld: 

\begin{figure}[H]
    \centering
    \includegraphics[width=0.5\textwidth]{functie-2degraad5.png}
    \caption{$y = x^2 - 6x + 8$}
\end{figure}

\subsubsection{Derdegraadsfunctie}

\begin{theorem}[Derdegraadsfunctie]
\begin{equation}
    \begin{aligned}
        f(x) = ax^3 + bx^2 + cx + d
        (a \neq 0)
    \end{aligned}
\end{equation}


\end{theorem}

\subsubsection{Exponenti"ele functie}

\begin{theorem}[Exponenti"ele functie]
\begin{equation}
    f(x) = a^{g(x)}
\end{equation}

Met grondtal $a \in \mathbb{R}_0^+ \backslash \{1\}$
\end{theorem}

\begin{figure}[H]
    \centering
    \includegraphics[width=0.3\textwidth]{functie-exponentieel.png}
    \includegraphics[width=0.3\textwidth]{functie-exponentieel2.png}
    \caption{}
\end{figure}


\begin{itemize}
    \item Betekenis van a: groeifactor
    \item Wanneer stijgend? 
    \item Wanneer dalend? 
    \item Nulpunten: 
    \item Vaststelling beeld functie
\end{itemize}



\begin{theorem}[Constante van Euler]

\begin{equation}
    \begin{aligned}
        e \approx 2.718281828\dots
    \end{aligned}
\end{equation}

$f(x) = e^x$ is een bijzondere exponenti"ele functie
\end{theorem}

\begin{figure}[H]
    \centering
    \includegraphics[width=0.5\textwidth]{functie-exponentieel3.png}
    \caption{Verschil tussen $2^x$, $3^x$ en $e^x$}
\end{figure}

\section{Exponenti"ele verbanden in data}

\subsection{Lineaire groei}

Kenmerkend:

\begin{itemize}
    \item Per tijdseenheid wordt hetzelfde getal \textbf{opgeteld}
    \item Grafiek is een rechte
    \item Algemene formule (N = aantal, t = tijd, b: beginhoeveelheid): 
    \begin{equation}
        N = a\cdot t + b
    \end{equation}
\end{itemize}

\begin{figure}[H]
    \centering
    \includegraphics[width=0.5\textwidth]{lineaire-groei.png}
    \caption{Lineaire groei}
\end{figure}

\subsection{Exponenti"ele groei}

Kenmerkend: 

\begin{itemize}
    \item Per tijdseenheid wordt de hoeveelheid met hetzelfde getal \textbf{vermenigvuldigd}
    \item Grafiek is een exponenti"ele functie
    \item \textbf{Algemene formule:} 
    \begin{equation}
        N = b \cdot g^t
    \end{equation}
\end{itemize}

\begin{figure}[H]
    \centering
    \includegraphics[width=0.5\textwidth]{exponentiele-groei.png}
    \caption{Exponenti"ele groei}
\end{figure}

\begin{figure}[H]
    \centering
    \includegraphics[width=0.5\textwidth]{voorbeeld-groei.png}
    \caption{Voorbeeld exponenti"ele groei met groeifactor $\approx 1.22$}
\end{figure}


\subsection{Van groeipercentage naar groeifactor}

De toename/afname wordt vaak ook procentueel uitgedrukt

\begin{itemize}
    \item Een jaarlijkse toename van $14.6\%$
    \item Een jaarlijkse afname van $14.6\%$
\end{itemize}

\begin{theorem}[Groeifactor]
De groeifactor is de factor die per tijdseenheid wordt vermenigvuldigd met de vorige waarde.
\end{theorem}

\subsubsection{Percentage naar factor}

\begin{equation}
g = \frac{p + 100}{100}\%
\end{equation} 

\begin{figure}[H]
    \centering
    \includegraphics[width=0.5\textwidth]{percentage-naar-factor.png}
    \caption{Van groeipercentage naar groeifactor}
\end{figure}


\subsubsection{Factor naar percentage}

\begin{figure}[H]
    \centering
    \includegraphics[width=0.35\textwidth]{factor-naar-percentage.png}
    \caption{Van groeifactor naar groeipercentage}
\end{figure}

\begin{figure}[H]
    \centering
    \includegraphics[width=0.5\textwidth]{groeifactoren.png}
    \caption{Let op: hier gebeuren vaak fouten bij het omrekenen}
\end{figure}

\subsection{Voorbeeld}

Een hoeveelheid groeit exponentieel. Na 5u is $N = 82$ en na 12u is $N = 246$.

Stel de formule van N op.

\textbf{Oplossing}

\begin{equation}
N = b \cdot g^t
\end{equation}

\underline{Stap 1: groeifactor berekenen per tijdseenheid:}

\begin{center}
$$
\left.
    \begin{array}{lll}
        \text{Na 5u}  & \rightarrow & N = 82 \\
        \text{Na 12u} & \rightarrow & N = 246 \\
    \end{array}
\right \} \Delta = 7u \rightarrow 164
$$


Groeifactor voor 7 uren: $\frac{246}{82} = 3$

Groeifactor voor 1 uur: $3^{1/7} \approx 1.170$
\end{center}

\underline{Stap 2: 1 punt nemen waarvan we N weten:}

\begin{center}

Gekozen punt: $(5, 82)$

$82 = b \cdot (1.170)^5$

$\Leftrightarrow b = \frac{82}{1.170}^5 \approx 37$

$\Leftrightarrow N = 37 \cdot 1.170^t$
\end{center}

\subsection{Belangrijke maten voor exponenti"ele toename}


\begin{theorem}[Verdubbelingstijd]
De verdubbelingstijd is de nodige tijd tot de hoeveelheid verdubbeld is.

De verdubbelingstijd $t$ kan je berekenen met:
\begin{equation}
g^t = 2
\end{equation}
\end{theorem}

\textbf{Oefening}

De populatie neemt toe met $8.3\%$ per jaar. Bereken de verdubbelingstijd:

\begin{center}
$g^t = 2$

$\Leftrightarrow (1.083)^t = 2$

$\Leftrightarrow \log(1.083^t) = \log(2)$

$\Leftrightarrow t \cdot \log(1.083) = \log(2)$

$\Leftrightarrow t = \frac{\log(2)}{\log(1.083)}$

$\Leftrightarrow t = 8.69\ jaar$

\end{center}


\begin{theorem}[Halveringstijd]
De halveringstijd is de nodige tijd tot de hoeveelheid gehalveerd is.

De halveringstijd $t$ kan je berekenen met:
\begin{equation}
g^t = 1/2
\end{equation}
\end{theorem}

\subsubsection{Oefening: Combinatie van groeifactoren?}

Een hoeveelheid neemt eerst 5 jaar lang met vast percentage (*) toe, 
om daarna nog 3 jaar met 10\% per jaar toe te nemen. Na 8 jaar is
de totale hoeveelheid verdubbeld.

(*) Bereken het jaarlijkse groeipercentage in de eerste 5 jaren.

\textbf{Oplossing}

We weten:

\begin{itemize}
    \item Eerste 5 jaar: toename met vast percentage
    \item Volgende 3 jaar: toename met 10\% (= factor van 1.1)
    \item Na 8 jaar: hoeveelheid verdubbeld (= factor van 2)
\end{itemize}

\begin{center}
$g^5 \cdot 1.1^3 = 2$

We moeten $g$ vinden:

$\Leftrightarrow g^5 = \frac{2}{1.1^3}$

$\Leftrightarrow g = \sqrt[5]{\frac{2}{1.1^3}}$
\end{center}

\section{Belangrijke functies met betrekking tot machine learning}

\subsection{Logistische groei}

\subsubsection{Voorbeeld}

Startsituatie: een bos (bv $\text{10km}^2$) waarin een konijnenepidemie uitbreekt.
Boswachter houdt de populatie van de konijnen bij. Wat stelt hij vast?

De groei van de populatie verloopt volgens een typisch patroon (niet exponentieel):

\begin{figure}[H]
    \centering
    \includegraphics[width=0.5\textwidth]{logistische-groei-vs-exponentieel.png}
    \caption{De rode lijn is de bovengrens}
\end{figure}

\subsubsection{De groei}

= de mate van toename

\begin{itemize}
    \item Hangt af van hoeveel er al zijn tegenover hoeveel er nog bij kunnen
    \item Heel sterke verandering bij start, op het einde heel kleine verandering
    \item Hangt dus ook af van de tijd
\end{itemize}

\begin{theorem}[De logistische groei]
De logistische groei is de mate van toename, afhankelijk van hoeveel er nog bij kan en hoeveel er al is

\begin{equation}
    \frac{\text{Hoeveel er nog bij kan}}{\text{Hoeveel er al is}} = B \cdot g^t
\end{equation}

\begin{itemize}
    \item t = de tijd,
    \item B en g = constanten
\end{itemize}


\end{theorem}

\subsubsection{Functievoorschrift}

\begin{equation}
y = \frac{G}{1 + B\cdot g^t}
\end{equation}

\begin{itemize}
    \item t = de tijd
    \item B en constanten
    \item G = bovengrens
\end{itemize}

\begin{figure}[H]
    \centering
    \includegraphics[width=0.5\textwidth]{logistische-groei.png}
    \caption{Grafiek logistische groei met G = 800}
\end{figure}

\subsubsection{Voorbeeld}

Het aantal vissen in een meer is gegeven door:

\begin{center}
    $N = \frac{2500}{1 + 5.5 \cdot 0.74^t}$
\end{center}

waarbij N = aantal vissen, t = tijd

\textbf{Beredeneer}: Wanneer bereiken we het 'verzadigingsniveau'

Als t heel groot is:

\begin{itemize}
    \item Dan wordt $0.74^t \approx 0$
    \item Dan wordt $5.5 \cdot 0.74^t \approx 0$
    \item Dan wordt $N \approx 2500$
    \item $\Rightarrow$ Het meer is `verzadigd'
\end{itemize}

\subsubsection{Algemene wiskundige notatie van een logistische functie}

\begin{theorem}[Logistische functie]
De wiskundige notatie voor een logistische functie is:

\begin{equation}
    f(x) = \frac{c}{1 + a\cdot b^x}
\end{equation}

met a,b,c constanten waarbij de constante c de belangrijkste is:

c drukt uit wat de maximumwaarde kan zijn
\end{theorem}

\begin{figure}[H]
    \centering
    \includegraphics[width=0.5\textwidth]{logistische-functie-wiskundig.png}
    \caption{}
\end{figure}

\subsection{Regression analysis}

Regressieanalyse:

\begin{itemize}
    \item Is er een (voorspellend) verband tussen 2 variabelen
    \item Heeft de ene variabele een invloed op de andere variabele
\end{itemize}

\begin{figure}[H]
    \centering
    \includegraphics[width=0.5\textwidth]{regressie.png}
    \caption{Regressieanalyse}
\end{figure}

\begin{figure}[H]
    \centering
    \includegraphics[width=0.5\textwidth]{regressie2.png}
    \caption{Lineaire vs niet-lineaire samenhang}
\end{figure}

\subsubsection{Regressiemodel}

Zoeken naar een model dat uitkomst (2 mogelijkheden) voorspelt mbv inputwaardes.
Elke inputwaarde heeft een zeker belang (gewicht)

\begin{figure}[H]
    \centering
    \includegraphics[width=0.5\textwidth]{regressiemodel.png}
    \caption{3 inputs met elk een bepaald gewicht, die een uitkomst zoekt (2 mogelijkheden)}
\end{figure}

(TODO slide 12)

\end{document}
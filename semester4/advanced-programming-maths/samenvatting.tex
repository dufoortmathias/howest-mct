\documentclass{article}

\usepackage[dutch]{babel}
\usepackage[margin=3cm]{geometry}
\usepackage{graphicx}
\usepackage{float}
\usepackage{caption}
\usepackage{hyperref}
\usepackage{amsmath}
\usepackage{wrapfig}
\usepackage[parfill]{parskip}

% fonts
\usepackage[T1]{fontenc}
\usepackage{helvet}
\renewcommand{\familydefault}{\sfdefault}

\graphicspath{{img/}}

% theorem environment
\usepackage{amssymb}

\newtheorem{theorem}{Definitie}[section]

\usepackage{enumitem}

\newenvironment{thmenum}
 {\begin{enumerate}[label=\upshape\bfseries(\roman*)]}
 {\end{enumerate}}


% code
\usepackage{minted}
\usepackage{upquote}
\usepackage{color}

\begin{document}

\begin{titlepage}
    \author{Tuur Vanhoutte}
    \title{Advanced Programming \& Maths}
\end{titlepage}

\pagenumbering{gobble}
\maketitle
\newpage
\tableofcontents
\newpage

\pagenumbering{arabic}

\section{Basisfuncties in de wiskunde}

\subsection{Functies}

\begin{theorem}[Re"ele functie]
Een re"ele functie is een relatie in $\mathbb{R}$ waarbij elke waarde $x$ hoogstens één beeldwaarde $f(x)$ heeft
\end{theorem}

\begin{figure}[H]
    \centering
    \includegraphics[width=0.5\textwidth]{reele-functie.png}
    \caption{Voorbeelden re"ele functies}
\end{figure}

\begin{theorem}
Voor elke functie geldt: er bestaat een \dots
    \begin{thmenum}
        \item \dots domein van de functie (domain)
        \item \dots beeld van de functie (range)
        \item \dots functievoorschrift van de functie
    \end{thmenum}
\end{theorem}

\begin{figure}[H]
    \centering
    \includegraphics[width=0.5\textwidth]{functie-domain-range.png}
    \caption{Domein, bereik, functievoorschrift}
\end{figure}

$f: \mathbf{domein} \rightarrow \mathbf{bereik}: x \rightarrow y = f(x)$

$f: \mathbb{R} \rightarrow \mathbb{R} : x \rightarrow y = x^3 - 4x$

\begin{theorem}
    Elke functie kan nulpunten hebben.
\end{theorem}

\begin{figure}[H]
    \centering
    \includegraphics[width=0.5\textwidth]{functie-nulpunten.png}
    \caption{$y=-x^3 + 4x$}
\end{figure}

Verloop van een functie wordt via een tekenschema verduidelijkt:

\begin{figure}[H]
    \centering
    \includegraphics[width=0.5\textwidth]{functie-tekenschema.png}
    \caption{Tekenschema}
\end{figure}

\subsection{Veelterm en veeltermfuncties}

\begin{theorem}[Veelterm]
\begin{equation}
    \begin{aligned}
        A(x) = a_nx^n + a_{n-1}x^{n-1} + a_{n-2}x^{n-2} + ... + a_{2}x^{2} + a_1x + a_0\\
        (a_n,a_{n-1},...,a_2,a_1,a_0 \in \mathbb{R})
    \end{aligned}
\end{equation}

\end{theorem}

\begin{theorem}[Veeltermfunctie]
\begin{equation}
    \begin{aligned}
        f(x) = a_nx^n + a_{n-1}x^{n-1} + a_{n-2}x^{n-2} + ... + a_{2}x^{2} + a_1x + a_0\\
        Graad\ van\ veelterm = n\ als\ a_n \neq 0
    \end{aligned}
\end{equation}
\end{theorem}

\subsection{Bijzondere veeltermfuncties}

\begin{itemize}
    \item Constante functie: $f(x) = 4$
    \item Lineaire functie: $f(x) = 4$
    \item Tweedegraadsfunctie: $f(x) = 3x^2 + 2x + 1$
    \item Derdegraadsfunctie: $f(x) = 5x^3 - 3x^2 + 2x - 1$
    \item Exponenti"ele functie: $f(x) = 2^x$
    \item Logaritmische functie: $(fx) = log_2(x)$
\end{itemize}

\subsubsection{Constante functie}

\begin{figure}[H]
    \centering
    \includegraphics[width=0.3\textwidth]{functie-constant.png}
    \caption{$y=4$}
\end{figure}

\subsubsection{Lineaire functie}

\begin{theorem}[Lineaire functie]
    \begin{equation}
        f(x) = ax + b
    \end{equation}


Voorbeeld: $f(x) = 3x + 6$
\end{theorem}

\begin{itemize}
    \item Betekenis van a: de richtingsco"effici"ent (rico)
    \item Betekenis van b: het snijpunt met de y-as
    \item Nulpunt: $f(x) = 0 \\ \Leftrightarrow 3x + 6 = 0 \\ \Leftrightarrow 3x = -6  \\ \Leftrightarrow x = -2$
\end{itemize}

\begin{figure}[H]
    \centering
    \includegraphics[width=0.3\textwidth]{functie-lineair2.png}
    \caption{Meerdere evenwijdige lineaire functies}
\end{figure}

Evenwijdige rechten als: als $a_1 = a_2$

Loodrechte rechten als: als $a_1 \cdot a_2 = -1$

\subsubsection{Tweedegraadsfunctie}

\begin{theorem}
\begin{equation}
    \begin{aligned}
        f(x) = ax^2 + bx + c,\\
        (a \neq 0)
    \end{aligned}
\end{equation}


\end{theorem}

\begin{figure}[H]
    \centering
    \includegraphics[width=0.2\textwidth]{functie-2degraad.png}
    \caption{$f(x) = x^2 - 2x - 3$}
\end{figure}

\begin{itemize}
    \item Betekenis van a: positief $\Rightarrow$ dalparabool, negatief $\Rightarrow$ bergparabool
    \item Nulpunten: via de discriminant berekenen:
\end{itemize}

\begin{theorem}[Discriminant]
Bij een tweedegraadsvergelijking is de discriminant:

\begin{equation}
    D = b^2 - 4ac
\end{equation}
\end{theorem}

\begin{itemize}
    \item Geval 1: $D > 0 \Rightarrow$ de functie heeft 2 nulpunten
    \item Geval 2: $D = 0 \Rightarrow$ de functie heeft 1 nulpunt
    \item Geval 3: $D < 0 \Rightarrow$ de functie heeft géén nulpunten
\end{itemize}

\begin{figure}[H]
    \centering
    \includegraphics[width=0.4\textwidth]{functie-2degraad3.png}
    \includegraphics[width=0.3\textwidth]{functie-2degraad2.png}
    \caption{De discriminant toont de nulpunten}
\end{figure}

\textbf{Nulpunten berekenen:} 

\begin{equation}
    x_{1,2} = \frac{-b \pm \sqrt{D}}{2a}
\end{equation}

\begin{figure}[H]
    \centering
    \includegraphics[width=0.5\textwidth]{functie-2degraad4.png}
    \caption{Symmetrieas: $x = \frac{-b}{2a}$}
\end{figure}

Voorbeeld: 

\begin{figure}[H]
    \centering
    \includegraphics[width=0.5\textwidth]{functie-2degraad5.png}
    \caption{$y = x^2 - 6x + 8$}
\end{figure}

\subsubsection{Derdegraadsfunctie}

\begin{theorem}[Derdegraadsfunctie]
\begin{equation}
    \begin{aligned}
        f(x) = ax^3 + bx^2 + cx + d
        (a \neq 0)
    \end{aligned}
\end{equation}


\end{theorem}

\subsubsection{Exponenti"ele functie}

\begin{theorem}[Exponenti"ele functie]
\begin{equation}
    f(x) = a^{g(x)}
\end{equation}

Met grondtal $a \in \mathbb{R}_0^+ \backslash \{1\}$
\end{theorem}

\begin{figure}[H]
    \centering
    \includegraphics[width=0.3\textwidth]{functie-exponentieel.png}
    \includegraphics[width=0.3\textwidth]{functie-exponentieel2.png}
    \caption{}
\end{figure}


\begin{itemize}
    \item Betekenis van a: groeifactor
    \item Wanneer stijgend? 
    \item Wanneer dalend? 
    \item Nulpunten: 
    \item Vaststelling beeld functie
\end{itemize}



\begin{theorem}[Constante van Euler]

\begin{equation}
    \begin{aligned}
        e \approx 2.718281828\dots
    \end{aligned}
\end{equation}

$f(x) = e^x$ is een bijzondere exponenti"ele functie
\end{theorem}

\begin{figure}[H]
    \centering
    \includegraphics[width=0.5\textwidth]{functie-exponentieel3.png}
    \caption{Verschil tussen $2^x$, $3^x$ en $e^x$}
\end{figure}

\section{Exponenti"ele verbanden in data}

\subsection{Lineaire groei}

Kenmerkend:

\begin{itemize}
    \item Per tijdseenheid wordt hetzelfde getal \textbf{opgeteld}
    \item Grafiek is een rechte
    \item Algemene formule (N = aantal, t = tijd, b: beginhoeveelheid): 
    \begin{equation}
        N = a\cdot t + b
    \end{equation}
\end{itemize}

\begin{figure}[H]
    \centering
    \includegraphics[width=0.5\textwidth]{lineaire-groei.png}
    \caption{Lineaire groei}
\end{figure}

\subsection{Exponenti"ele groei}

Kenmerkend: 

\begin{itemize}
    \item Per tijdseenheid wordt de hoeveelheid met hetzelfde getal \textbf{vermenigvuldigd}
    \item Grafiek is een exponenti"ele functie
    \item \textbf{Algemene formule:} 
    \begin{equation}
        N = b \cdot g^t
    \end{equation}
\end{itemize}

\begin{figure}[H]
    \centering
    \includegraphics[width=0.5\textwidth]{exponentiele-groei.png}
    \caption{Exponenti"ele groei}
\end{figure}

\begin{figure}[H]
    \centering
    \includegraphics[width=0.5\textwidth]{voorbeeld-groei.png}
    \caption{Voorbeeld exponenti"ele groei met groeifactor $\approx 1.22$}
\end{figure}


\subsection{Van groeipercentage naar groeifactor}

De toename/afname wordt vaak ook procentueel uitgedrukt

\begin{itemize}
    \item Een jaarlijkse toename van $14.6\%$
    \item Een jaarlijkse afname van $14.6\%$
\end{itemize}

\begin{theorem}[Groeifactor]
De groeifactor is de factor die per tijdseenheid wordt vermenigvuldigd met de vorige waarde.
\end{theorem}

\subsubsection{Percentage naar factor}

\begin{equation}
g = \frac{p + 100}{100}\%
\end{equation} 

\begin{figure}[H]
    \centering
    \includegraphics[width=0.5\textwidth]{percentage-naar-factor.png}
    \caption{Van groeipercentage naar groeifactor}
\end{figure}


\subsubsection{Factor naar percentage}

\begin{figure}[H]
    \centering
    \includegraphics[width=0.35\textwidth]{factor-naar-percentage.png}
    \caption{Van groeifactor naar groeipercentage}
\end{figure}

\begin{figure}[H]
    \centering
    \includegraphics[width=0.5\textwidth]{groeifactoren.png}
    \caption{Let op: hier gebeuren vaak fouten bij het omrekenen}
\end{figure}

\subsection{Voorbeeld}

Een hoeveelheid groeit exponentieel. Na 5u is $N = 82$ en na 12u is $N = 246$.

Stel de formule van N op.

\textbf{Oplossing}

\begin{equation}
N = b \cdot g^t
\end{equation}

\underline{Stap 1: groeifactor berekenen per tijdseenheid:}

\begin{center}
$$
\left.
    \begin{array}{lll}
        \text{Na 5u}  & \rightarrow & N = 82 \\
        \text{Na 12u} & \rightarrow & N = 246 \\
    \end{array}
\right \} \Delta = 7u \rightarrow 164
$$


Groeifactor voor 7 uren: $\frac{246}{82} = 3$

Groeifactor voor 1 uur: $3^{1/7} \approx 1.170$
\end{center}

\underline{Stap 2: 1 punt nemen waarvan we N weten:}

\begin{center}

Gekozen punt: $(5, 82)$

$82 = b \cdot (1.170)^5$

$\Leftrightarrow b = \frac{82}{1.170}^5 \approx 37$

$\Leftrightarrow N = 37 \cdot 1.170^t$
\end{center}

\subsection{Belangrijke maten voor exponenti"ele toename}


\begin{theorem}[Verdubbelingstijd]
De verdubbelingstijd is de nodige tijd tot de hoeveelheid verdubbeld is.

De verdubbelingstijd $t$ kan je berekenen met:
\begin{equation}
g^t = 2
\end{equation}
\end{theorem}

\textbf{Oefening}

De populatie neemt toe met $8.3\%$ per jaar. Bereken de verdubbelingstijd:

\begin{center}
$g^t = 2$

$\Leftrightarrow (1.083)^t = 2$

$\Leftrightarrow \log(1.083^t) = \log(2)$

$\Leftrightarrow t \cdot \log(1.083) = \log(2)$

$\Leftrightarrow t = \frac{\log(2)}{\log(1.083)}$

$\Leftrightarrow t = 8.69\ jaar$

\end{center}


\begin{theorem}[Halveringstijd]
De halveringstijd is de nodige tijd tot de hoeveelheid gehalveerd is.

De halveringstijd $t$ kan je berekenen met:
\begin{equation}
g^t = 1/2
\end{equation}
\end{theorem}

\subsubsection{Oefening: Combinatie van groeifactoren?}

Een hoeveelheid neemt eerst 5 jaar lang met vast percentage (*) toe, 
om daarna nog 3 jaar met 10\% per jaar toe te nemen. Na 8 jaar is
de totale hoeveelheid verdubbeld.

(*) Bereken het jaarlijkse groeipercentage in de eerste 5 jaren.

\textbf{Oplossing}

We weten:

\begin{itemize}
    \item Eerste 5 jaar: toename met vast percentage
    \item Volgende 3 jaar: toename met 10\% (= factor van 1.1)
    \item Na 8 jaar: hoeveelheid verdubbeld (= factor van 2)
\end{itemize}

\begin{center}
$g^5 \cdot 1.1^3 = 2$

We moeten $g$ vinden:

$\Leftrightarrow g^5 = \frac{2}{1.1^3}$

$\Leftrightarrow g = \sqrt[5]{\frac{2}{1.1^3}}$
\end{center}

\section{Belangrijke functies met betrekking tot machine learning}

\subsection{Logistische groei}

\subsubsection{Voorbeeld}

Startsituatie: een bos (bv $\text{10km}^2$) waarin een konijnenepidemie uitbreekt.
Boswachter houdt de populatie van de konijnen bij. Wat stelt hij vast?

De groei van de populatie verloopt volgens een typisch patroon (niet exponentieel):

\begin{figure}[H]
    \centering
    \includegraphics[width=0.5\textwidth]{logistische-groei-vs-exponentieel.png}
    \caption{De rode lijn is de bovengrens}
\end{figure}

\subsubsection{De groei}

= de mate van toename

\begin{itemize}
    \item Hangt af van hoeveel er al zijn tegenover hoeveel er nog bij kunnen
    \item Heel sterke verandering bij start, op het einde heel kleine verandering
    \item Hangt dus ook af van de tijd
\end{itemize}

\begin{theorem}[De logistische groei]
De logistische groei is de mate van toename, afhankelijk van hoeveel er nog bij kan en hoeveel er al is

\begin{equation}
    \frac{\text{Hoeveel er nog bij kan}}{\text{Hoeveel er al is}} = B \cdot g^t
\end{equation}

\begin{itemize}
    \item t = de tijd,
    \item B en g = constanten
\end{itemize}


\end{theorem}

\subsubsection{Functievoorschrift}

\begin{equation}
y = \frac{G}{1 + B\cdot g^t}
\end{equation}

\begin{itemize}
    \item t = de tijd
    \item B en constanten
    \item G = bovengrens
\end{itemize}

\begin{figure}[H]
    \centering
    \includegraphics[width=0.5\textwidth]{logistische-groei.png}
    \caption{Grafiek logistische groei met G = 800}
\end{figure}

\subsubsection{Voorbeeld}

Het aantal vissen in een meer is gegeven door:

\begin{center}
    $N = \frac{2500}{1 + 5.5 \cdot 0.74^t}$
\end{center}

waarbij N = aantal vissen, t = tijd

\textbf{Beredeneer}: Wanneer bereiken we het 'verzadigingsniveau'

Als t heel groot is:

\begin{itemize}
    \item Dan wordt $0.74^t \approx 0$
    \item Dan wordt $5.5 \cdot 0.74^t \approx 0$
    \item Dan wordt $N \approx 2500$
    \item $\Rightarrow$ Het meer is `verzadigd'
\end{itemize}

\subsubsection{Algemene wiskundige notatie van een logistische functie}

\begin{theorem}[Logistische functie]
De wiskundige notatie voor een logistische functie is:

\begin{equation}
    f(x) = \frac{c}{1 + a\cdot b^x}
\end{equation}

met a,b,c constanten waarbij de constante c de belangrijkste is:

c drukt uit wat de maximumwaarde kan zijn
\end{theorem}

\begin{figure}[H]
    \centering
    \includegraphics[width=0.5\textwidth]{logistische-functie-wiskundig.png}
    \caption{}
\end{figure}

\subsection{Regression analysis}

Regressieanalyse:

\begin{itemize}
    \item Is er een (voorspellend) verband tussen 2 variabelen
    \item Heeft de ene variabele een invloed op de andere variabele
\end{itemize}

\begin{figure}[H]
    \centering
    \includegraphics[width=0.5\textwidth]{regressie.png}
    \caption{Regressieanalyse}
\end{figure}

\begin{figure}[H]
    \centering
    \includegraphics[width=0.5\textwidth]{regressie2.png}
    \caption{Lineaire vs niet-lineaire samenhang}
\end{figure}


\subsubsection{Lineair regressiemodel}

Enkelvoudige vorm: 

\begin{itemize}
    \item 1 inputwaarde x
    \item via lineaire functie $h_{\theta}(x) = \theta_0 + \theta_1x$
\end{itemize}

\begin{figure}[H]
    \centering
    \includegraphics[width=0.3\textwidth]{lineair-regressiemodel.png}
    \caption{}
\end{figure}

\begin{itemize}
    \item Aan de hand van de opgestelde functie doe je een voorspelling
    \item \textbf{Doel:} een zo goed mogelijke lineaire functie opstellen
    \item $\Rightarrow$ zoektocht naar de beste $\theta_0$ en $\theta_1$
\end{itemize}


\subsubsection{Logistisch regressiemodel}

Logistische regressie = \textbf{Classificatie-algoritme}

Zoeken naar een model dat uitkomst (2 mogelijkheden) voorspelt mbv inputwaardes.
Elke inputwaarde heeft een zeker belang (gewicht).

\begin{figure}[H]
    \centering
    \includegraphics[width=0.5\textwidth]{regressiemodel.png}
    \caption{3 inputs met elk een bepaald gewicht, die een uitkomst zoekt (2 mogelijkheden)}
\end{figure}

Vereenvoudiging:  

\begin{itemize}
    \item 1 inputwaarde x
    \item Logistische functie $p = \frac{1}{1 + e^{-(b_0+b_1x)}}$
\end{itemize}

Uitkomst:

\begin{itemize}
    \item de persoon slaagt als $h_{\theta}(x) \geq 0.5$
    \item de persoon slaagt niet als $h_{\theta}(x) < 0.5$
\end{itemize}

\begin{figure}[H]
    \centering
    \includegraphics[width=0.3\textwidth]{logistisch-regressiemodel.png}
    \caption{1 inputwaarde x, met twee  uitkomsten}
\end{figure}

\subsubsection{Lineair vs logistisch regressiemodel}

\begin{figure}[H]
    \centering
    \includegraphics[width=0.5\textwidth]{lineair-vs-logistisch-regression-model.png}
    \caption{Hoe dichter p tegen 1, hoe zekerder het model is}
\end{figure}

Welk model gaat het snelst naar 0 en 1? 

\begin{itemize}
    \item Het logistische model
    \item Daarom is het logistische model beter voor classificate: je splitst de groep op in 2
\end{itemize}

\subsubsection{Meerdere inputfactoren}

Zelfde redenering:

\begin{itemize}
    \item Meerdere inputwaardes $x_1, x_2, \dots$
    \item Gebruik $\theta_0 + \theta_1x_1 + \theta_2x_2 + \dots$
\end{itemize}

\begin{figure}[H]
    \centering
    \includegraphics[width=0.3\textwidth]{logistische-regressie-meerdere-inputfactoren.png}
    \caption{Regressiemodel met meerdere inputfactoren}
\end{figure}


\subsection{Softmax functie}

Doelstelling:

\begin{itemize}
    \item Model dat in staat is om data te gaan categoriseren
    \item Hoe?
    \begin{itemize}
        \item $\Rightarrow$ Met behulp van verschillende inputvariabelen en bijhorende parameters 
    \end{itemize}
\end{itemize}

\begin{figure}[H]
    \centering
    \includegraphics[width=0.35\textwidth]{softmax.png}
    \includegraphics[width=0.4\textwidth]{softmax2.png}
    \caption{Categoriseren met de softmax functie}
\end{figure}

\subsubsection{Kansen}

Kans dat de toestand tot groep A behoort:

\begin{itemize}
    \item $\theta_{A,0} + \theta_{A,1}x_1 + \theta_{A,2}x_2$
    \item Voorbeeld: $0.01 + 0.1x_1 + 0.1x_2$
\end{itemize}

Kans dat de toestand tot groep B behoort:

\begin{itemize}
    \item $\theta_{B,0} + \theta_{B,1}x_1 + \theta_{B,2}x_2$
    \item Voorbeeld: $0.1 + 0.2x_1 + 0.2x_2$
\end{itemize}

Kans dat de toestand tot groep C behoort:

\begin{itemize}
    \item $\theta_{C,0} + \theta_{C,1}x_1 + \theta_{C,2}x_2$
    \item Voorbeeld: $0.1 + 0.3x_1 + 0.3x_2$
\end{itemize}

\subsubsection{Model}

Het softmax-model berekent de mate van zekerheid dat een toestand tot een bepaalde categorie behoort.

vb: volgende quotiënt drukt uit hoe zeker hij is dat (z1, z2) tot categorie A behoort:

$\frac{e^{\theta_{A,0} + \theta_{A,1}z_1 + \theta_{A,2}z_2}}{e^{\theta_{A,0} + \theta_{A,1}z_1 + \theta_{A,2}z_2} + e^{\theta_{B,0} + \theta_{B,1}z_1 + \theta_{B,2}z_2} + e^{\theta_{C,0} + \theta_{C,1}z_1 + \theta_{C,2}z_2} }$

(analoog voor categorie B en C: vervang de teller)

\begin{figure}[H]
    \centering
    \includegraphics[width=0.5\textwidth]{softmax-voorbeeld.png}
    \caption{Betekenis: het model is 29\% zeker dat (0.1, 0.5) tot categorie A behoort. Bereken zelf als oefening voor B en C}
\end{figure}

\subsubsection{Wiskundig}

Het gebruikte model wordt via volgende wiskundige formule algemeen beschreven:

\begin{equation}
\frac{e^{x_k}}{\sum_{i=1}^n e^{x_i}}
\end{equation}

waarbij:

\begin{itemize}
    \item $x_k = \theta_{k,0} + \theta_{k,1}x_1 + \theta_{k,2}x_2 + \dots + \theta_{k,m}x_m$
    \item n = aantal groepen
    \item m = het aantal meetcriteria
\end{itemize}

\subsection{Logistic regression cost function}

Het model:

\begin{equation}
h_{\theta}(x) = \frac{1}{1 + e^{-\theta^{\tau}x}}
\end{equation}

waarbij:

\begin{itemize}
    \item $\theta^{\tau}x = \theta_0 + \theta_1x_1 + \theta_2x_2$
    \item $h_{\theta}$ drukt uit wat de kans is dat voor opgegeven $x_1$ en $x_2$ de waarneming tot 1 groep behoort
    \item $x_1$ en $x_2$ zijn de inputwaardes
    \item $\theta_1$ en $\theta_2$ zijn gewichten (hoe belangrijk is de input)
    \item \textbf{Doel: } vinden van de beste gewichten zodat de voorspelling == de werkelijkheid
\end{itemize}

\subsubsection{Success meten}

\textbf{Stel:} je maakt een logistisch regressiemodel die bepaalt of een object een groene appel of een tennisbal is. 

\begin{itemize}
    \item Bepalen van de kostenfunctie $J(\theta)$ met als doel deze zo laag mogelijk te brengen
    \item kost = afwijking tegenover de werkelijke situatie
    \item werkelijkheid kan 2 situaties zijn:
    \begin{itemize}
        \item Indien de werkelijkheid een groene appel is $\Rightarrow y = 1$ 
        \item Indien de werkelijkheid géén groene appel is $\Rightarrow y = 0$ 
    \end{itemize}
\end{itemize}

Hoe ziet zo'n kostfunctie er dan uit?

\begin{figure}[H]
    \centering
    \includegraphics[width=0.4\textwidth]{logistische-regressie-kost.png}
    \caption{Als $y = 1$ en $y = 0$}
\end{figure}

\begin{figure}[H]
    \centering
    \includegraphics[width=0.3\textwidth]{logistische-regressie-kost2.png}
    \caption{}
\end{figure}

\begin{figure}[H]
    \centering
    \includegraphics[width=0.4\textwidth]{logistische-regressie-kost3.png}
    \caption{}
\end{figure}

Hoe brengen we 2 mogelijke situaties in 1 functie samen?


(TODO: slide 24 - 32)


\section{Pandas library}

\subsection{Inleiding}

\begin{itemize}
    \item Doelstelling:
    \begin{itemize}
        \item Nut van de pandas library kunnen situeren
        \item Data-analyse: basisbewerkingen
    \end{itemize}
    \item Pandas = `Python Data Analysis Library'
    \item Pandas bouwt op de NumPy library
    \item Officiële website: \url{https://pandas.pydata.org/}
    \item Goede start: \url{http://pandas.pydata.org/pandas-docs/stable/10min.html} 
\end{itemize}

\subsubsection{Welke data verwerken?}

\begin{itemize}
    \item csv-files
    \item txt-files
    \item Excel-files
    \item Databases
\end{itemize}

\subsection{Pandas.core}

Beschikbare datastructuren:

\begin{itemize}
    \item Series (1D)
    \item DataFrame (2D)
    \item Panel (3D)
\end{itemize}

\subsection{Series}

Bestemd voor 1-dimensionale data: 

`a one-dimension labeled array capable of holding any data'


\begin{itemize}
    \item Subklasse van numpy-ndarray
    \item Data: elk soort datatype
    \item Geordende index
    \item Duplicaten mag (maar niet optimaal)
\end{itemize}

\begin{figure}[H]
    \centering
    \includegraphics[width=0.2\textwidth]{panda-series.png}
    \caption{Elk element heeft een index}
\end{figure}

\subsection{DataFrame}

Bestemd voor meer-dimensionale data

\begin{itemize}
    \item Subklasse van numpy-ndarray
    \item Elke kolom kan ander datatype hebben
    \item Rij en kolom index
    \item Grootte wijzigbaar (invoegen/verwijderen van rijen en kolommen)
\end{itemize}

\begin{figure}[H]
    \centering
    \includegraphics[width=0.95\textwidth]{panda-dataframe0.png}
    \caption{DataFrames maken uit Python lists en dictionaries}
\end{figure}


\begin{figure}[H]
    \centering
    \includegraphics[width=0.25\textwidth]{panda-dataframe.png}
    \includegraphics[width=0.35\textwidth]{panda-dataframe2.png}
    \caption{Elk element heeft een rij en kolom}
\end{figure}


\subsubsection{Select data from DataFrame}

\textbf{Via operator [] selecteer je een kolom:}

\begin{figure}[H]
    \centering
    \includegraphics[width=0.3\textwidth]{panda-dataframe-data.png}
    \caption{Voorbeeld DataFrame}
\end{figure}


\begin{minted}{python}
# selecteer de kolom Jan uit dataframe
df['Jan']

# analoog: elke kolom is dus een attribuut van dataframe
df.Jan

# returnwaarde: Series-object
0   150
1   200
2   50
Name: Jan, dtype: int64
\end{minted}

\textbf{Via operator [[]] selecteer je een kolom \& krijg je een dataframe terug:}

\begin{minted}{python}
>> df[['Jan']]
# returns:
    Jan
0   150
1   200
2   50

>> df[['Jan', 'Mar']]
# returns:
    Jan     Mar
0   150     140
1   200     215
2   50      95
\end{minted}

\textbf{Via de operator [] en met een conditie:}

\begin{figure}[H]
    \centering
    \includegraphics[width=0.3\textwidth]{panda-dataframe-data2.png}
    \caption{Voorbeeld DataFrame}
\end{figure}

\begin{minted}{python}
>> df[df.Jan > 60]
# returns:
    Feb     Jan     Mar     account
0   200     150     140     Jones LLC
1   210     200     215     Alpha Co

>> df[np.logical_and(df.Jan > 100, df.Feb <= 200)]
# returns:
    Feb     Jan     Mar     account
0   200     150     140     Jones LLC

>> df[df.account.str.startswith(‘Alpha’)]
# test deze eens zelf uit als oefening :)
\end{minted}

\subsubsection{Veelgebruikte commandos bij dataframes}

\begin{minted}{python}
df.shape            # geeft de dimensie als een tuple terug
df.info()           # oplijsting van de aanwezige kolommen
df.head([aantal])   # eerste vijf/aantal rijen
df.tail([aantal])   # laatste vijf/aantal rijen
df.index            # geef de index-kolom weer
df.colums           # geef de kolomnamen weer
df.describe()       # geef snel overzicht van statistische data
df.T                # transponeer data (rij -> kol, kol -> rij)
df.sort_index()     # sorteer op basis van index
df.sort_values()    # sorteren op één of meerdere kolommen
\end{minted}


\subsection{Loc vs iloc}

\subsubsection{iloc}

= Integer-location based indexing / selection by position

\begin{itemize}
    \item Nut: selecteren van rijen en kolommen via rij/kolomnummer
    \item Syntax: data.iloc[<row>, <column>]
    \item Returnwaarde:
    \begin{itemize}
        \item Indien 1 \textbf{rij} $\Rightarrow$ series-object
        \item Indien meerdere \textbf{rijen}: $\Rightarrow$ dataframe-object
        \item 1 of meerdere \textbf{kolommen}: $\Rightarrow$ dataframe-object 
    \end{itemize}
\end{itemize}

\textbf{iloc-voorbeelden:}

\begin{minted}{python}
# Rows:
data.iloc[0] # first row of data frame
data.iloc[1] # second row of data frame
data.iloc[-1] # last row of data frame

# Columns:
data.iloc[:,0] # first column of data frame
data.iloc[:,1] # second column of data frame
data.iloc[:,-1] # last column of data frame

data.iloc[0:5] # first five rows of dataframe

# first two columns of data frame with all rows
data.iloc[:, 0:2] 

# 1st, 4th, 7th, 25th row + 1st 6th 7th columns.
data.iloc[[0,3,6,24], [0,5,6]]  

# first 5 rows and 5th, 6th, 7th columns of data frame
data.iloc[0:5, 5:8]

\end{minted}

\subsubsection{loc}

= label based indexing / selection

\begin{itemize}
    \item Nut: selecteren van rijen en kolommen via label / via conditionele look-up
    \item Syntax: data.loc[<row>, <column>]
    \item Returnwaarde:
    \begin{itemize}
        \item Indien 1 \textbf{rij/kol} $\Rightarrow$ series-object
        \item Indien meerdere \textbf{rijen}: $\Rightarrow$ dataframe-object
        \item 1 of meerdere \textbf{kolommen}: $\Rightarrow$ dataframe-object 
    \end{itemize}
\end{itemize}


\textbf{loc-voorbeelden:}

\begin{figure}[H]
    \centering
    \includegraphics[width=0.4\textwidth]{pandas-loc-voorbeeld.png}
    \caption{Voorbeeld dataframe}
\end{figure}


\begin{minted}{python}
reviews.loc[:2, "score"] # return type = 

reviews.loc[:2, ["score", "title"]] # return type = 

# select column "score" where value of index <= 5
reviews.loc[:5, "score"]

# select columns "country" and "cars_per_cap" where rowindex is "US" or "RU"
cars.loc[ ["US","RU"] , ["country","cars_per_cap"]]

# select columns "country" and "cars_per_cap" where rowindex is from "US" to "RU"
cars.loc[ "US " : "RU " , ["country","cars_per_cap"]]

# selectie rijen hoeft niet altijd op basis van row-index te zijn
# select columns "country" and "drives_right", voor de landen 'Japan' en 'India'
cars.loc[ cars.country.isin( ['Japan', 'India'] ) , ['country','drives_right']]
\end{minted}

\subsection{Plotten met pandas}

\subsubsection{Dataframe plotten}

\begin{minted}{python}
# print(cars[['country', 'cars_per_cap']])
# werkwijze 1:
cars[['country', 'cars_per_cap']].plot(kind='bar', legend=True)
# werkwijze 2:
cars.plot(x='country', y='cars_per_cap', kind='bar', legend=True)

plt.show()
\end{minted}

\begin{figure}[H]
    \centering
    \includegraphics[width=0.4\textwidth]{pandas-plotting.png}
    \caption{Resultaat}
\end{figure}

\subsubsection{Series plotten}

\begin{minted}{python}
# werkwijze 1:
plt.plot(cars['cars_per_cap'])
# werkwijze 2:
plt.plot(cars['cars_per_cap'].plot(color='r', legend=True))

plt.show()
\end{minted}

\begin{figure}[H]
    \centering
    \includegraphics[width=0.4\textwidth]{pandas-plotting2.png}
    \caption{Resultaat}
\end{figure}

\subsection{Demo: Iris Dataset}

(Zie DemoPandas.zip op Leho voor de code)

\begin{figure}[H]
    \centering
    \includegraphics[width=0.5\textwidth]{pandas-demo.png}
    \caption{Een iris bestaat uit petals \& sepals, met elk hun breedte en lengte}
\end{figure}

\begin{figure}[H]
    \centering
    \includegraphics[width=0.5\textwidth]{pandas-demo2.png}
    \caption{Plotten van de beschikbare data (demo5.py)}
\end{figure}

\begin{figure}[H]
    \centering
    \includegraphics[width=0.3\textwidth]{pandas-demo3.png}
    \caption{Vergelijken van de lengtes en breedtes adhv box-plots (demo6.py)}
\end{figure}

\begin{minted}{python}
# filteren van data
result_check = iris['Species'] == 'Iris-setosa'
# print(type(result_check)) == TODO
filtered_setosa = iris.loc[result_check, :]
# of in 1 lijn:
filtered_setosa = iris.loc[iris['Species'] == 'Iris-setosa', :]
\end{minted}

\begin{figure}[H]
    \centering
    \includegraphics[width=0.3\textwidth]{pandas-demo4.png}
    \includegraphics[width=0.4\textwidth]{pandas-demo5.png}
    \caption{Frequentiediagram voor de Iris-setosa soort (demo7.py)}
\end{figure}

\subsection{Complexe bewerkingen}

\begin{figure}[H]
    \centering
    \includegraphics[width=0.8\textwidth]{pandas-complexe-bewerkingen.png}
    \caption{Reshaping data: change the layout of a data set}
\end{figure}

(komen we later nog op terug)

\section{}


\end{document}